\documentclass[11pt]{article}
\usepackage{ifthen}
\usepackage{soul}
\usepackage{fullpage, prettyref}
\usepackage{fancyhdr}
\usepackage{palatino}
\usepackage{graphicx}
\usepackage{marvosym}
\usepackage{ifthen}
\usepackage{algorithm}

\usepackage[noend]{algpseudocode}
\algrenewcomment[1]{\(\triangleright\) {\small{\color{blue} #1}}}
\algnewcommand{\LineComment}[1]{\State \(\triangleright\) \emph{\color{blue} #1}}
\newcommand{\Lim}{\lim}


\usepackage{amsmath, amssymb, amsthm}
\usepackage{url}
\usepackage{pstricks,pst-node,pst-text}
\usepackage{boxedminipage}
%\usepackage{hyperref}
\usepackage{wrapfig}
\usepackage{ifthen}
\usepackage{color,xcolor}
\usepackage{transparent}

\usepackage{varwidth}

\usepackage{a4,geometry}
\usepackage{graphicx}
\usepackage{amsmath,amssymb,amsthm,mathtools}
\usepackage{paralist}
\usepackage{bm}
\usepackage{xspace}
\usepackage{url}
%\usepackage{fullpage, prettyref}
\usepackage{boxedminipage}
\usepackage{wrapfig}
\usepackage{ifthen}
\usepackage{color}
\usepackage{xcolor}
\usepackage{framed}
\usepackage{mdframed}
\usepackage[pagebackref,colorlinks=true,pdfpagemode=none,urlcolor=blue,linkcolor=blue,citecolor=violet,pdfstartview=FitH]{hyperref}
\usepackage{fullpage}
\usepackage{enumitem}


\theoremstyle{definition}
\newtheorem{theorem}{Theorem}
\newtheorem{lemma}{Lemma}
\newtheorem{claim}{Claim}
\newtheorem{corollary}{Corollary}
\newtheorem{definition}{Definition}
\newtheorem{proposition}{Proposition}
\newtheorem{fact}{Fact}
\newtheorem{conjecture}{Conjecture}
\newtheorem{example}{Example}
\newtheorem{question}{Question}
\newtheorem{exercise}{Problem}
\newtheorem{assumption}{Assumption}
\newtheorem{observation}{Observation}

\newcommand{\comment}[1]{\textsl{\small[#1]}\marginpar{\tiny\textsc{To Do!}}}
\newcommand{\ignore}[1]{}

\def\eps{\varepsilon}
\def\bar{\overline}
\def\floor#1{\lfloor {#1} \rfloor}
\def\ceil#1{\lceil {#1} \rceil}
\def\script#1{\mathcal{#1}}

\def\plus{{\tt (+)}}
\def\2plus{{\tt (++)}}
\def\3plus{{\tt (+++)}}
\def\4plus{{\tt (++++)}}
\def\5plus{{\tt (+++++)}}

\def\opt{{\tt opt}}
\def\alg{{\tt alg}}

\def\bv{\mathbf{v}}

\newenvironment{proofof}[1]{\smallskip\noindent{\bf Proof of #1:}}%
{\hspace*{\fill}$\Box$\par}
\setlength{\oddsidemargin}{0pt}
\setlength{\evensidemargin}{0pt}
\setlength{\textwidth}{6.5in}
\setlength{\topmargin}{0in}
\setlength{\textheight}{8.5in}
\newlength{\algobox}
\setlength{\algobox}{6.5in}

\newcommand{\cI}{\mathcal{I}}
\newcommand{\cA}{\mathcal{A}}

\newcommand{\Exp}{\mathbf{Exp}}
\renewcommand{\Pr}{\mathbf{Pr}}

\colorlet{shadecolor}{blue!10}
\newcommand{\remark}[1]{\begin{mdframed}[backgroundcolor=yellow!10,topline=false,bottomline=false,leftline=false,rightline=false] 
		{\bf Remark: }{\em #1} 
\end{mdframed}}

\newcommand{\todo}[1]{\marginpar{\Large \WritingHand} \begin{mdframed}[backgroundcolor=teal!10,topline=false,bottomline=false,leftline=false,rightline=false] 
		{\bf Question:}{\em #1}
\end{mdframed}}

\newcommand{\problem}[2]{\begin{mdframed}[backgroundcolor=blue!05,topline=false,bottomline=false,leftline=false,rightline=false] 
		{\underline{{\sc #1}}} \\
		#2 
\end{mdframed}}

\newcommand{\thingstoadd}[1]{\begin{mdframed}[backgroundcolor=red!10,topline=false,bottomline=false,leftline=false,rightline=false] 
		{\bf Things to add: }{\em #1} 
\end{mdframed}}

%make one of the 1's into 0 to hide solutions
\ifthenelse{\equal{0}{1}}
{
	\newcommand{\solution}[1]{\noindent {\color{blue} {\bf Solution:}} {\color{gray} {#1}}}
	\newcommand{\rubric}[1]{\ignore{#1}}%{\noindent {\color{red} {\bf Rubric:}} {\texttransparent{0.5} {\bf #1}}}
	\newcommand{\hint}[1]{\ignore{#1}}
	\newcommand{\announce}[1]{}
}
{
	\newcommand{\solution}[1]{\ignore{#1}}
	\newcommand{\rubric}[1]{\ignore{#1}}
	\newcommand{\announce}[1]{\ignore{#1}}
	\newcommand{\hint}[1]{\noindent   {\texttransparent{0.5}{Hint: #1}}}
}

\begin{document}
	\begin{center}
		{\bf \Large CS 49/149: 21st Century Algorithms (Fall 2018): Problem Set 0}
	\end{center}
\small{
The problems below are not for submission. Rather, it indicates the mathematical maturity that a student should be comfortable with.
That is, if a student feels that they can {\em tackle} these problems and is confident that given enough time can either solve or make considerable progress, then they should have no problems in the math that they will face in CS 49/149. 

Also, we will be using some facts established below throughout the course.
}
\vspace{1ex}
\hrule height 1pt
\vspace{1ex}
%\hrule height 1pt
\noindent

\begin{center}
	{\Large \bf Instructions}
\end{center}
\begin{itemize}
	\item {\em Presentation}:
	In most problems below you will be asked to describe an algorithm and analyze it. Please keep in mind the following things
	\begin{itemize}[noitemsep]
		\item 
		Explain the idea behind your algorithm in English. 
		If the idea is clear from the description, you need not even write the pseudocode description. If you do write pseudocode, please provide ``comments'' to explain what your variables, assertions, etc mean.
		
		\item 
		Many times we will ask you to justify your algorithm. You then need to give enough arguments to convince the reader of your algorithm.
		If an argument needs a formal proof, provide it.
		
		\item Always provide the running time of the algorithms you describe. Unless mentioned, this should be in the Big-Oh notation.
	\end{itemize}
	

\item {\em General small print:} Please submit all homework electronically in PDF format ideally typeset using LaTeX. 
%You need to submit only the problems above the line. 
Please try to be concise -- as a rule of thumb do not take more than 1 (LaTeX-ed) page for a solution. 


\item {\em Collaboration Policy:} You are allowed to discuss with other students but are {\em not allowed} to exchange full solutions.
At the beginning of each problem you must write who you discussed with, and what way did the person help you or you help them. This is important. 
If you did not talk with anyone about any of the problems, mention this at the beginning of the homework. You may not consult any solutions on the Internet or from previous years' assignments, whether they are student- or faculty-generated.


\item {\em Extra Credit Problems:} Sometimes I will assign problems for extra-credit. 
Please note the extra-credit problems cannot ``make up'' for losses in the problems for credit.
Often, but not always, they will be harder, and will not be rigorously graded (sometimes it may be a problem we don't 
know the solution to). So {\em try these only after} doing the for-credit problems.

\end{itemize}	

\announce{}
\thispagestyle{empty}
\newpage
\setcounter{page}{1}
\begin{exercise}
	Prove the following inequalities
	\begin{enumerate}
		\item (\Coffeecup )
		For any real $x$,  prove $e^x \geq 1 + x$.
		
		\solution{
		There are many ways to prove this. One way is to define a function $g(x) = e^x - (1+x)$ and to prove this is $\geq 0$.
		We see that $\dot{g}(x) = e^x - 1$ which is $\geq 0$ when $x \geq 0$ and $< 0$ when $x < 0$. 
		Thus, $g(x)$ is increasing when $x\geq 0$ implying $g(x) \geq g(0) =0$ for all $x\geq 0$; and that 
		$g(x)$ is decreasing when $x < 0$ implying $g(x) \geq g(0) = 0$ for all $x < 0$. 
		}
		
		\item (\Coffeecup ) 
		For any real $x$, prove $\frac{e^x + e^{-x}}{2} \leq e^{x^2}$.
		
		\solution{
		By definition of $e^x$, the LHS is 
		\[
		1 + x^2/2! + x^4/4! + x^6/6! + \cdots
		\]
		while the RHS is
		\[
		1 + x^2 + x^4/2! + x^6/3! + \cdots 
		\]
		Term by term the RHS is bigger (with equality at $x = 0$).
		}
		
		\item (\Coffeecup \Coffeecup)
		For any real $x < 0.5$,  $e^x \leq 1 + x + x^2$.
		
		\solution{
		Similar idea to the one above: define $h(x) = e^x - (1 + x + x^2)$.
		Observe $\dot{h}(x) = e^x - (1+ 2x)$. 
		}
		
	\end{enumerate}
\end{exercise}

\vspace{2ex}

\begin{exercise}
	Prove the following
	\begin{enumerate}
		\item (\Coffeecup\Coffeecup) For any $n > 0$, $\sum_{i=1}^n 1/i = \Theta(\log n)$.
		
		\solution{
			The trick is to compare with the integration $\int_{1}^{n} dx/x$.
			Define the function defined for $x\geq 0$:
			\[
			f(x) := 1/i \textrm{ for all $x\in (i-1,i]$ }, ~~~ \textrm{integer $i \geq 1$}
			\]

			First, observe
			\[
			\textrm{For all $x\ge 0$,} \qquad\frac{1}{x+1} \leq f(x) \leq \frac{1}{x}
			\]
			Second, observe that for any $1 \leq  m < n$, $\sum_{i=m}^n 1/i = \int_{m-1}^n f(x) dx$. 
			Thus, the first inequality above implies
			\[
			\sum_{i=1}^n 1/i = \int_{0}^n f(x)dx \geq \int_{0}^{n} \frac{dx}{x+1} = \ln(n+1)
			\]
			while the second inequality implies
			\[
			\sum_{i=1}^n 1/i = 1 + \sum_{i=2}^n 1/i = 1 + \int_{1}^n f(x)dx \leq 1 + \int_{1}^{n} \frac{dx}{x} = 1 + \ln n
			\]
			This finishes the proof.
			}
		
		\item (\Coffeecup) For any $n > 0$, $\sum_{i=1}^n 1/i^2 = \Theta(1)$.
		
			\solution{
				Similar ideas as above. Clearly the LHS is $\ge 1$; for the upper bound define $f(x) := 1/i^2$ for $x \in (i-1,i]$, and observe $f(x) \leq 1/x^2$ for all $x \geq 0$.
				Then use
				\[
				\sum_{i=1}^n 1/i^2 = 1 + \int_{1}^n f(x)dx \leq 1 + \int_{1}^n dx/x^2 \leq 2
				\]
				In fact, one can do better than above ... the answer is well known to converge to $\pi^2/6$.
				
			}
		
	\end{enumerate}
\end{exercise}

\vspace{2ex}

\begin{exercise} 
	
Stirling's approximation states that for any natural number $n$,
\[
\sqrt{2\pi}\cdot  \left(n^{n + \frac{1}{2}} e^{-n}\right) ~\leq n! ~\leq e \cdot \left(n^{n + \frac{1}{2}} e^{-n} \right)
\]	

(\Coffeecup \Coffeecup) Use this to prove  ${n \choose n/2}/2^n = \Theta(1/\sqrt{n})$ (assume $n$ is even).
	
	\solution{
		The LHS, by definition, is $\frac{n!}{(n/2)!(n/2)!2^n}$. For an upper bound, we can use Stirling's approximation to get that the LHS is at most
		\[
		\frac{e\cdot \left(n^{n + \frac{1}{2}} e^{-n} \right)}{2\pi \cdot \left((n/2)^{(n/2) + \frac{1}{2}} e^{-(n/2)}\right)^2 \cdot 2^n  } 
		= \frac{e\cdot \left(n^{n + \frac{1}{2}} \right)}{2\pi \cdot \left((n/2)^{n + 1} \right) \cdot 2^n  }  
		= \frac{e}{\pi \sqrt{n}}  
		\]
		The RHS is similar.
	}
	
(\Coffeecup \Coffeecup \Coffeecup) ({\bf Extra Credit:}) In fact prove that for any constant $c$, 
	\[
	{n \choose \frac{n}{2} \pm c\sqrt{n}}/2^n = \Theta(1/\sqrt{n})
	\]


\end{exercise}

\vspace{2ex}
\begin{exercise}(Probability Basics)
	\begin{enumerate}
		\item (\Coffeecup) Prove for any {\em non-negative} random variable $Z$ and $t > 0$,
		\[
		\Pr[Z > t] \quad < \quad \frac{\Exp[Z]}{t}
		\]
		This is called Markov's inequality, or sometimes an {\em averaging argument}.
		
		\solution{
			\[
			\Exp[Z] = \sum_{z} z\cdot \Pr[Z = z] =  \sum_{z > t} z\cdot \Pr[Z = z] + \sum_{0\leq z \le t}  z\cdot \Pr[Z = z] 
			\]
			The first summand is $\geq t\cdot \Pr[Z \geq t]$ and the second summand is $\geq 0$. This proves the inequality.
		}
	
		\item (\Coffeecup) Prove for any random variable $Z$ and $t > 0$, 
		\[
		\Pr[|Z - \Exp[Z]| > t] \quad \leq \quad \frac{\mathbf{Var}[Z]}{t^2}
		\]
		This is called Chebyshev's inequality or the second-moment inequality.
		
		\solution{
			Simply apply Markov's inequality on the positive random variable $(Z - \Exp[Z])^2$.
			Observe,
			\[
\Pr[|Z - \Exp[Z]| > t] = \Pr[(Z - \Exp[Z])^2 > t^2] \leq \Exp((Z - \Exp Z)^2)/t^2 = \mathbf{Var}Z/t^2
			\]
		}
	
		\item 
		Now we establish a much stronger inequality for sums of {\em independent variables}. This is {\em super important} and such bounds are called Chernoff-Hoeffding-Azuma bounds. Knowing the result is important, but is also important to know how things are proved. We do this in steps.
		
		For $1\leq i\leq n$, let $X_i$ be a random variable which is $+1$ with probability $1/2$ and $-1$ with probability $1/2$.
		These are mutually {\em independent} random variables (recall what this is). Let $Z = X_1 + \cdots + X_n$ be the sum of these random variable.
		
		\begin{enumerate}
			\item (\Coffeecup) What is $\Exp[Z]$?
			
			\solution{$0$ since $\Exp[X_i] = 0$}
			
			\item (\Coffeecup) What is $\mathbf{Var}[Z]$? What upper bound does the Chebyshev inequality say about $\Pr[Z > t]$ for $t > 0$?
			
			\solution{
				$\mathbf{Var}[Z] = \sum_{i=1}^n \mathbf{Var}[X_i]$ since the $X_i$'s are independent.
				Now, the variance of each $X_i$ is $1$. Thus, Chebyshev tells us
				\[
				\Pr[Z > t] \leq \Pr[|Z - \Exp[Z]| > t] \leq n/t^2
				\]
			}
			
			\item (\Coffeecup)  Now note that $\Pr[Z > t]$ is at most $\Pr[e^{\lambda Z} > e^{\lambda t}]$ for any $\lambda \ge 0$. 
			Also note that $e^{\lambda Z}$ is a positive random variable.
			Apply Markov's inequality on it.
			
			\solution{
			We get 
			\[
			\Pr[Z > t] \leq e^{-\lambda t} \Exp[e^{\lambda Z}] 
			\]	
		}
			
			\item (\Coffeecup \Coffeecup) What is $\Exp[e^{\lambda Z}]$? Recall what $Z$ is, and recall that the $X_i$'s are independent. 
			Recall the relation between $\Exp[f(X)g(Y)]$ and $\Exp[f(X)]\cdot \Exp[g(Y)]$ for deterministic function $f$ and $g$.
		
		\solution{
		\[
		\Exp[e^{\lambda Z}] = \Exp[e^{\lambda(\sum_{i=1}^n X_i)}] = \Exp[\prod_{i=1}^n e^{\lambda X_i}] = \prod_{i=1}^n \Exp[e^{\lambda X_i}]
		\]
		where the last equality follows from the independence of $X_i$'s.
		Finally, we note that $\Exp[e^{\lambda X_i}] = \frac{e^\lambda + e^{-\lambda}}{2}$. Thus, we get
		\[
		\Exp[e^{\lambda Z}] = \left(\frac{e^\lambda + e^{-\lambda}}{2}\right)^n
		\]
}
			
			
			\item (\Coffeecup \Coffeecup) Now can you optimize $\lambda$ so that the RHS of what you get from part 3 (and putting in part 4's calculations) is minimized? What is the final answer you get?
			
	\solution{
	Putting in part 4's calculation into part 3 gives (and also using problem 1, part 2)
	\[
	\Pr[Z > t] \quad \leq e^{-\lambda t} \cdot e^{\lambda^2 n}
	\]
	To minimize the RHS, we must choose $\lambda$ such that $-\lambda t + \lambda^2 n$ is minimized. This occurs at $\lambda = t/2n$ giving
	\[
	\Pr[Z > t] \quad \leq e^{-t^2 / 4n}
	\]
	}		
		\end{enumerate}
	
%\item 
%\begin{enumerate}
%	\item (\Coffeecup)
%	Prove for any random variable $Y$ that $\Exp[Y^2] \geq (\Exp[Y])^2$.
%	\solution{
%Indeed the difference between LHS and RHS is the variance, which is $\Exp[(Y - \Exp[Y])^2]$ which is $\geq 0$.
%	}
%	\item (\Coffeecup) Let  $X_i \in \{\pm 1\}$ with probability $1/2$ and let $Z = \sum_{i=1}^n X_i$. 
%	Prove that $\Exp[|Z|] \leq \sqrt{n}$.
%\end{enumerate}
	\end{enumerate}
\end{exercise}

\end{document}