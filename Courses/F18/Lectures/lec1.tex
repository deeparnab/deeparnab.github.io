\documentclass[11pt]{article}
\usepackage{ifthen}
\usepackage{soul}
\usepackage{fullpage, prettyref}
\usepackage{fancyhdr}
\usepackage{palatino}
\usepackage{graphicx}
\usepackage{marvosym}
\usepackage{ifthen}
\usepackage{algorithm}
\usepackage[noend]{algpseudocode}
\algrenewcomment[1]{\(\triangleright\) {\tiny{#1}}}
\algnewcommand{\LineComment}[1]{\State \(\triangleright\) \emph{\color{blue} #1}}
\newcommand{\Lim}{\lim}
\newcommand{\Exp}{\mathbf{Exp}}

\usepackage{amsmath, amssymb, amsthm}
\usepackage{url}
\usepackage{fullpage, prettyref}
\usepackage{pstricks,pst-node,pst-text}
\usepackage{boxedminipage}
\usepackage{hyperref}
\usepackage{wrapfig}
\usepackage{ifthen}
\usepackage{mdframed}
\usepackage{color,xcolor}
\usepackage{transparent}
\usepackage{enumitem}
\usepackage{varwidth}
\theoremstyle{definition}
\newtheorem{theorem}{Theorem}
\newtheorem{lemma}{Lemma}
\newtheorem{claim}{Claim}
\newtheorem{corollary}{Corollary}
\newtheorem{definition}{Definition}
\newtheorem{proposition}{Proposition}
\newtheorem{fact}{Fact}
\newtheorem{example}{Example}
\newtheorem{exercise}{Problem}
\newtheorem{drill}{Exercise}
\newtheorem{assumption}{Assumption}
\newtheorem{observation}{Observation}

\newcommand{\comment}[1]{\textsl{\small[#1]}\marginpar{\tiny\textsc{To Do!}}}
\newcommand{\ignore}[1]{}

\def\eps{\varepsilon}
\def\bar{\overline}
\def\floor#1{\lfloor {#1} \rfloor}
\def\ceil#1{\lceil {#1} \rceil}
\def\script#1{\mathcal{#1}}

\def\plus{{\tt (+)}}
\def\2plus{{\tt (++)}}
\def\3plus{{\tt (+++)}}
\def\4plus{{\tt (++++)}}
\def\5plus{{\tt (+++++)}}

\def\opt{{\tt opt}}
\def\alg{{\tt alg}}

\def\bv{\mathbf{v}}

\newenvironment{proofof}[1]{\smallskip\noindent{\bf Proof of #1:}}%
{\hspace*{\fill}$\Box$\par}
\setlength{\oddsidemargin}{0pt}
\setlength{\evensidemargin}{0pt}
\setlength{\textwidth}{6.5in}
\setlength{\topmargin}{0in}
\setlength{\textheight}{8.5in}
\newlength{\algobox}
\setlength{\algobox}{6.5in}


%make one of the 1's into 0 to hide solutions

\begin{document}
	\begin{center}
		{\bf \Large CS 49/149: 21st Century Algorithms (Fall 2018): Lecture 1}\\ 
		Date: 13th September, 2018 \\
		Topic: The Experts Problem \\
		Scribe: Deeparnab Chakrabarty \\
		{\em Disclaimer: These notes have not gone through scrutiny and in all probability contain errors. Please email errors to deeparnab@dartmouth.edu.}
	\end{center}
\hrule height 2pt
\vspace{3ex}
\def\loss{\mathsf{loss}}
\section{The Experts Problem}

Suppose you want to predict if it is going to rain or not. Unfortunately, you have no idea of meteorology. But you have $m$ friends who are {\em experts} (that is, they have a PhD; it doesn't mean they are infallible) who are willing to tell you their opinions. Your goal is to use their opinions to make a good prediction.

Let's formalize. We denote $e_i(t) \in \{-1,1\}$ to be expert $i$'s prediction of whether it will rain or not on the $t$th day. The index $i$ ranges from $1$ to $m$.
You observe these $e_i(t)$'s, and then you need to make a prediction $a(t) \in \{-1,1\}$. Then the $t$th day unfolds, and you get to see $r(t) \in \{-1,1\}$ whether it actually rained or not.
If $a(t)$ doesn't match $r(t)$, say you pay a dollar -- this is your {\em loss} $\ell(t)$ on day $t$. Thus, formally, $\ell(t) = (1 - a(t)\cdot r(t))/2$. We want to devise a strategy, an algorithm, to make the total loss incurred, $\loss := \sum_{t=1}^T \ell(t)$, as small as possible. This is the experts problem. \smallskip

Note that this is not your usual computation problem with a fixed input and output. It has a flavor of a game that one is playing with nature, and in a sense nature (who is responsible for $r(t)$) gets to ``play second''. That is, our algorithm's decision has to be made under uncertainty of the outcome. Whenever this happens, it is not clear what ``the best algorithm'' even means.
What is a good benchmark to compare against?


Well, since our algorithm is deriving information from the experts, perhaps we should compare our losses with those of the $m$ experts. In particular, at time $t$ define $\ell_i(t)$ to be $1$ if expert $i$ makes a mistake at time $i$, and let $\loss_i := \sum_{t=1}^T \ell_i(t)$.
But our experts can have differing qualities -- which one should we choose our benchmark as? Well, it is clear that the ``best expert'' (the one with the smallest $\loss_i$) is the most stringent among these benchmarks; let us see if we can design an algorithm which is as good as the best expert. 

\paragraph{The case of the perfect expert.}

Let's start with the case there is one ``perfect expert''. That is, there is some $i^*$ with $\loss_{i^*} = 0$. If we knew who this perfect expert was, then our strategy is clear -- predict whatever $i^*$ is predicting. Can we quickly recognize this expert? The answer is yes, and the algorithm is simple. It maintains a candidate set $A$ of active experts; any expert not in $A$ has made some mistake in the past and thus cannot be the perfect expert. Initially $A = [m]$, and we wish to whittle this set down fast. The {\sc Majority} algorithm is this: always predict what the majority of the experts in $A$ are predicting (if $|A|$ is even, break ties arbitrarily). Every time our algorithm makes a mistake, that is, everytime $\loss$ increments by $+1$, we know that at least $|A|/2$ experts are whittled out. Thus, the maximum number of mistakes {\sc Majority} makes is $\log_2 m$.

\begin{theorem}
In the case of the perfect expert, the {\sc Majority} algorithm finds this expert making $\log_2 m$ mistakes.
\end{theorem}

\todo{
Can any algorithm always find the perfect expert making $\ll \log_2 m$ queries?
}

However, perfect experts are mythical. What if we were only promised an expert who is correct $99\%$ of the time? How would {\sc Majority} perform? Well, not too well as defined since the near-perfect expert could make a mistake on the first day along with the majority and be immediately whittled out. How should we fix this? 

Idea 1: Suppose we knew we were playing for $T$ days. Keep taking {\sc Majority} till some expert makes $> 0.01T$ mistakes and then whittle him/her out. Two issues: one, it needs to know that some expert was as good as $99\%$ (how would we know that?). More seriously, it's not a good algorithm -- find an example where this algorithm fails badly.  \smallskip

Idea 2: When an expert makes a mistake, instead of whittling them out, just decrease their ``importance''. More precisely, every expert has an importance or a weight $w_i(t)$. Initially, all the weights are $1$; all experts are equal in our eyes. Each day, we go with the {\em weighted} majority. More precisely, we compute $\sum_{i=1}^m w_i(t)e_i(t)$ and predict its sign (with $0$ considered ``positive''). Upon receiving the truth, that is, $r(t)$, we {\em update} the weights as follows: we penalize every expert which makes a mistake at time $t$ by halving their weight. The full algorithm is described below.

\begin{mdframed}[backgroundcolor=blue!05,topline=false,bottomline=false,leftline=false,rightline=false] 
	\underline{\sc Weighted Majority}
	\begin{itemize}
		\item Maintain weights $w_i()$ for each expert with $w_i(1) = 1$ for all $i$.
		\item On days $t=1,\ldots, T$:
		\begin{itemize}
			\item We receive $e_i(t)$ from each $i$.
					\item For each prediction $\{+1,-1\}$ we calculate the total {\em weight} of experts predicting it, and go with whichever is larger.
			\item Then we receive $r(t)$. For every $i$ with $e_i(t) \neq r(t)$, we set
			\begin{equation}
			\label{eq:halving}
			w_i(t+1) = w_i(t)/2
			\end{equation}
			
		\end{itemize} 
	\end{itemize}
\end{mdframed}

\begin{theorem}
	If there exists an expert which makes at most $k$ mistakes, then the {\sc Weighted Majority} algorithm makes at most $2.41k + O(\log m)$ mistakes.
\end{theorem}

\begin{corollary}
	If there is an expert who is correct at least $99\%$ of the time, then if one plays for $T \gg \log m$ days, the {\sc Weighted Majority} is correct
	$\ge 96\%$ of the time.
\end{corollary}

\begin{proof} (of Theorem.)
	The analysis is similar in spirit to the analysis of the {\sc Majority} algorithm -- we focus on the times $t$ at which the algorithm makes a mistake.
	Consider such a time $t$ when $a(t) \neq r(t)$. By the algorithm's design we get to see at this $t$,
	\begin{equation}\label{eq:majority}
	\sum_{i: e_i(t) \neq r(t)} w_i(t) \geq 	\sum_{i: e_i(t) = r(t)} w_i(t) 
	\end{equation}
	Now note that for all experts in the LHS, their weight $w_i(t+1) = w_i(t)/2$ while for all experts in the right, $w_i(t+1) = w_i(t)$. Therefore,
	the {\em total weight} scales down.
	
	More precisely, let's define for any $t$,
	\[
	Z(t) := \sum_{i=1}^m w_i(t)
	\]
	For any time $t$ at which {\sc Weighted Majority} makes a mistake, \eqref{eq:majority} implies $\sum_{i: e_i(t) = r(t)} w_i(t) \leq Z(t)/2$.
	Furthermore, we get
	\begin{alignat}{4}
	Z(t+1) \qquad && = &~ \sum_{i: e_i(t) \neq r(t)} w_i(t+1) + \sum_{i: e_i(t) = r(t)} w_i(t+1) \notag\\
		   && = &~ \sum_{i: e_i(t) \neq r(t)} w_i(t)/2 + \sum_{i: e_i(t) = r(t)} w_i(t) \notag \\
		   && = &~ Z(t)/2 + \frac{1}{2} \sum_{i: e_i(t) = r(t)} w_i(t) \notag \\
		   && \leq &~ 3Z(t)/4 \notag
	\end{alignat}
	The above was for the times when our algorithm makes a mistake. What about the times when it doesn't? Well, since the weights never {\em increase}, we have the trivial inequality $Z(t+1) \leq Z(t)$.
	Thus, if our algorithm makes $\ell$ mistakes (that is, $\ell = \loss$), after $T$ rounds we get
	\begin{equation}\label{eq:ub}
	Z(T) \leq (3/4)^\ell \cdot Z(1) = (3/4)^\ell \cdot m
	\end{equation}
	
	So if our algorithm makes a lot of mistakes, the {\em potential} $Z$ falls rapidly. Why is this at all useful in comparing with the best expert's loss? 
	This is the second insight: if there is some expert making only $k$ mistakes, then his weight at the end is {\em at least} $(1/2)^k$. Even if all the other experts are duds making tons and tons of mistakes, this expert forces
	\begin{equation}\label{eq:lb}
	Z(T) \geq (0.5)^k
	\end{equation}
	Now rest is arithmetic. From \eqref{eq:ub} and \eqref{eq:lb} we get
	\[
	(1/2)^k \leq m\cdot (3/4)^\ell
	\]
	Taking logs base $2$ and swapping signs, we get
	\[
	k \geq - \log_2 m + \ell \log_2 (4/3)
	\]
	and then changing sides, we get
	\[
	\ell \leq \frac{1}{\log_2(4/3)}\cdot\left(k + \log_2 m\right)
	\]
	completing the proof.
\end{proof}

Can we do better? Well surely what was special about ``halving''. What if instead we scaled down by some other factor $(1- \eta)$. That is, consider {\sc Weighted Majority} where \eqref{eq:halving} is replaced by 
\[
w_i(t+1) = w_i(t) \cdot (1 - \eta)
\]

\begin{theorem}
		If there exists an expert which makes at most $k$ mistakes, then the {\sc Weighted Majority} algorithm with parameter $\eta$ makes at most $(2+\eta)k  + O\left(\frac{\log m}{\eta}\right)$ mistakes.
\end{theorem}
\begin{proof}
	The proof is similar to the one above with a little more arithmetic jugglery. The inequality corresponding to \eqref{eq:ub} becomes (check this!)
	\[
	Z(T) \leq (1 - \frac{\eta}{2})^\ell \cdot m
	\]
	Note that when $\eta = 1/2$, we get \eqref{eq:ub}. Similarly, \eqref{eq:lb} becomes (check this!)
	\[
	Z(T) \geq (1 - \eta)^k
	\]
	Therefore, together we get
	\[
	(1 - \eta)^k \leq m\cdot (1 - \eta/2)^\ell 
	\]
	which in turn implies
	\[
	k\ln(1-\eta) \leq \ln m + \ell \ln(1-\eta/2)
	\]
	Finally, we use the fact that for any $|x|\leq 1$, we have $-(x+x^2) \leq \ln(1-x) \leq -x$ giving us
	\[
	-k(\eta + \eta^2) \leq \ln m - \ell \eta/2
	\]
	Moving things around we get
	\[
	\ell \leq 2(1+\eta)k + \frac{2\ln m}{\eta}
	\]
\end{proof}

So we got the $2.41$ down to ``arbitrarily'' close to $2$. Can we do better? Turns out that deterministic algorithms can't do better.

\todo{
Show that given any deterministic algorithm can't get a better than factor $2$ approximation. More precisely, for any strategy and for any $T$, show
a collection of expert answers and ``truths'' such that the best expert makes $\leq \delta$ mistakes but the algorithm makes $2\delta$ mistakes.
In fact, this is true even with just {\em two} experts.
}

\subsection{Getting arbitrarily close with randomization}

Suppose now are algorithm is allowed to toss coins. That is, the quantity $a(t)$ is a random variable which $-1$ with a certain probability and $+1$ with the remainder.
The {\em expected} number of mistakes made by the algorithm at time $t$ is therefore the {\em probability} with which $a(t) \neq r(t)$. The total expected loss is the sum of these expectations (recall, linearity of expectation). 

What would be the ``natural'' algorithm if we allowed randomization? Note that in the {\sc Weighted Majority} if $50.001\%$ of the weight voted for $+1$ and $49.999\%$ voted for $-1$, the algorithm went with $+1$. Even with this slim lead. The ``fairer'' randomized algorithm that suggests itself is that we should say $+1$ with probability $50.001\%$ and $-1$ with the remainder probability. 
Turns out, this algorithm leads us arbitrarily close to the best expert!

A more convenient way of looking at the {\sc Randomized Weighted Majority} algorithm is by thinking of picking a random expert and going with their decision. That is, given the weights we pick an expert $i$ proportional to $w_i(t)$, and then predict $a(t) = e_i(t)$. Observe that this is the same fair algorithm described above. 

\begin{mdframed}[backgroundcolor=blue!05,topline=false,bottomline=false,leftline=false,rightline=false] 
	\underline{\sc Randomized Weighted Majority}
	\begin{itemize}
		\item Maintain weights $w_i()$ for each expert with $w_i(1) = 1$ for all $i$.
		\item On days $t=1,\ldots, T$:
		\begin{itemize}
			\item Select expert $i$ with probability $\propto w_i(t)$.
			\item Predict $a(t) = e_i(t)$.
			\item Then we receive $r(t)$. For every $i$ with $e_i(t) \neq r(t)$, we set
			\begin{equation}
			\label{eq:halving}
			w_i(t+1) = w_i(t)\cdot (1 - \eta)
			\end{equation}
			
		\end{itemize} 
	\end{itemize}
\end{mdframed}

\begin{theorem}
	If there is an expert making at most $k$ mistakes, then the {\em expected} number of mistakes made by the {\sc Randomized Weighted Majority} algorithm is 
	\[
	\Exp[\loss] \leq (1 + \eta)\cdot k + \frac{O(\log m)}{\eta}
	\]
\end{theorem}

\begin{proof}
	The proof idea is similar to the deterministic proof at a high level. As before, let $Z(t) := \sum_{i=1}^m w_i(t)$. Note we have \eqref{eq:lb} as is; to repeat
	\begin{equation}\label{eq:lb-rand}
	Z(T) \geq (1 - \eta)^k \geq e^{-k(\eta + \eta^2)}
	\end{equation}
	We have used the old inequality $\ln(1-x)\ge -x-x^2$ for all $|x|\leq 1$, above.
	
	Now the greedy rule is no longer true.  Let $\loss_t$ be the random variable indicating whether $a(t) \neq r(t)$. Note that,
	\begin{equation}
	\label{eq:exploss}
	\Exp[\loss_t] = \sum_{i: r(t) \neq e_i(t)} \Pr[i ~\textrm{selected}] = \frac{1}{Z(t)} \sum_{i: r(t) \neq e_i(t)} w_i(t)
	\end{equation}
	Now, note that the difference in the `potential' is
	\[
	Z(t+1) - Z(t) = -\eta \sum_{i: r(t) \neq e_i(t)} w_i(t) = -\eta Z(t)\Exp[\loss_t]
	\]
	giving us
	\[
	Z(t+1) \leq Z(t)\cdot\left(1 - \eta \Exp[\loss_t] \right) \leq Z(t)\cdot \exp\left(-\eta \Exp[\loss_t]\right)
	\]
	We have used $1+x\leq e^x$ for all $x$ above. Why? Because it allows us to ``telescope''.
	We  get
	\begin{equation}\label{eq:ub-rand}
	Z(T) \leq m\cdot \exp\left(-\eta \sum_{t=1}^T \Exp[\loss_t]\right) = m\cdot \exp\left(-\eta\cdot\Exp[\loss]\right)
	\end{equation}
	Taking natural logs on \eqref{eq:ub-rand} and \eqref{eq:lb-rand}, gives
	\[
	-(\eta + \eta^2)k \leq \ln m - \eta \Exp[\loss]
	\]
	The theorem follows by ``moving stuff around''.
\end{proof}

\end{document}

