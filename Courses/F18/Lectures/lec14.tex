\documentclass[11pt]{article}
\usepackage{ifthen}
\usepackage{soul}
\usepackage{fullpage, prettyref}
\usepackage{fancyhdr}
\usepackage{palatino}
\usepackage{graphicx}
\usepackage{marvosym}
\usepackage{ifthen}
\usepackage{algorithm}
\usepackage[noend]{algpseudocode}
\algrenewcomment[1]{\(\triangleright\) {\tiny{#1}}}
\algnewcommand{\LineComment}[1]{\State \(\triangleright\) \emph{\color{blue} #1}}
\newcommand{\Lim}{\lim}
\newcommand{\Exp}{\mathbf{Exp}}

\usepackage{amsmath, amssymb, amsthm}
\usepackage{url}
\usepackage{fullpage, prettyref}
\usepackage{pstricks,pst-node,pst-text}
\usepackage{boxedminipage}
\usepackage{hyperref}
\usepackage{wrapfig}
\usepackage{ifthen}
\usepackage{mdframed}
\usepackage{color,xcolor}
\usepackage{transparent}
\usepackage{enumitem}
\usepackage{varwidth}
\theoremstyle{definition}
\newtheorem{theorem}{Theorem}
\newtheorem{lemma}{Lemma}
\newtheorem{claim}{Claim}
\newtheorem{corollary}{Corollary}
\newtheorem{definition}{Definition}
\newtheorem{proposition}{Proposition}
\newtheorem{fact}{Fact}
\newtheorem{example}{Example}
\newtheorem{exercise}{Problem}
\newtheorem{drill}{Exercise}
\newtheorem{assumption}{Assumption}
\newtheorem{observation}{Observation}

\newcommand{\comment}[1]{\textsl{\small[#1]}\marginpar{\tiny\textsc{To Do!}}}
\newcommand{\ignore}[1]{}

\def\eps{\varepsilon}
\def\bar{\overline}
\def\floor#1{\lfloor {#1} \rfloor}
\def\ceil#1{\lceil {#1} \rceil}
\def\script#1{\mathcal{#1}}

\def\plus{{\tt (+)}}
\def\2plus{{\tt (++)}}
\def\3plus{{\tt (+++)}}
\def\4plus{{\tt (++++)}}
\def\5plus{{\tt (+++++)}}

\def\opt{{\tt opt}}
\def\alg{{\tt alg}}

\def\bv{\mathbf{v}}

\newenvironment{proofof}[1]{\smallskip\noindent{\bf Proof of #1:}}%
{\hspace*{\fill}$\Box$\par}
\setlength{\oddsidemargin}{0pt}
\setlength{\evensidemargin}{0pt}
\setlength{\textwidth}{6.5in}
\setlength{\topmargin}{0in}
\setlength{\textheight}{8.5in}
\newlength{\algobox}
\setlength{\algobox}{6.5in}



%make one of the 1's into 0 to hide solutions

\begin{document}
	\begin{center}
		{\bf \Large CS 49/149: 21st Century Algorithms (Fall 2018): Lecture 14}\\ 
		Date: 30th October, 2018 \\
		Topic: Streaming. Estimating the second moment -- two algorithms \\
		Scribe: Chao Chen \\
		{\em Disclaimer: These notes have not gone through scrutiny and in all probability contain errors. Please email errors to chao.chen.gr@dartmouth.edu.}
	\end{center}
\hrule height 2pt
\vspace{3ex}
\def\loss{\mathsf{loss}}
\section{Recap}
\subsection{Frequency Moment Estimation}
Think of we have a really large array A[1...n] coming in as a stream. Once an element leaves the stream, it is gone. Elements of the array are selected from the set ranging from 1 to m. Thus, A[i]$\in$\{1, 2, ..., m\}. And $f_j$ is the number of occurrences of j in the stream for j$\in$\{1, 2, ..., m\}. \\
We want to estimate $K^{th}$ moment:$$F_k = \sum_{j=1}^{m}f_j^k$$
\subsection{Estimation Algorithm}
We want to develop an algorithm which returns a random variable \textbf{Z} such that
\begin{itemize}
    \item 
    Unbiased:\\
    \centerline{$\mathbb{E}(Z)$ = "what we want"}
    \item 
    Error:\\
    \centerline{Additive Error:\ \ \ \ \  $\Pr[|Z-\mathbb{E}(Z)|\geq\epsilon]\leq \delta$\ \ \ \ \ \ \ \  (1)}
    \centerline{Multiplicative Error:\ \ \ \ \  
    $\Pr[Z\notin \mathbb{E}(Z)(1+\epsilon)]\leq \delta$\ \ \ \ \ \ \ \  (2)} 
\end{itemize}
From last class, we know if we want (1), we need $\mathrm{Var}(Z)\frac{1}{\epsilon^2}\ln{\frac{1}{\delta}}$ samples. If we want (2), we need $\frac{\mathrm{Var}(Z)}{\mathbb{E}(Z)^2}\frac{1}{\epsilon^2}\ln{\frac{1}{\delta}} \leq \frac{\mathbb{E}(Z^2)}{\mathbb{E}(Z)^2}\frac{1}{\epsilon^2}\ln{\frac{1}{\delta}} $ samples.

\section{Estimate $F_2$}
\subsection{Problem}
We want to estimate $F_2 = \sum_{j=1}^m f_j^2$ under the assumption that $n\approx m$. Consider this two cases:
\begin{enumerate}
    \item The elements have approximate equal frequency, i.e. for each j, $f_j \approx \frac{n}{m}$: $$F_2 = m\cdot(\frac{n}{m})^2 = \frac{n^2}{m} \approx n$$
    \item Some elements have large frequency while others are small, i.e for some j, $f_j$ is really large (we call this a "surprise factor"), $F_2$ will be really large.
\end{enumerate}

\subsection{Algorithm 1 -- Try 1}
\begin{mdframed}[backgroundcolor=blue!05,topline=false,bottomline=false,leftline=false,rightline=false] 
	\underline{\sc Try 1}
	\begin{itemize}
		\item Sample a j$\in$\{1, 2, ..., m\} u.a.r
		\item Count/Evaluate $f_j$
		\item return $Z=mf_j^2$
	\end{itemize}
\end{mdframed}
\textbf{Analysis:} We need to evaluate the bound of the number of samples Try1 need. Remember that it's bound by $\frac{\mathbb{E}(Z^2)}{\mathbb{E}(Z)^2}\frac{1}{\epsilon^2}\ln{\frac{1}{\delta}} $.\\
Calculate $\mathbb{E}(Z)$:
$$\mathbb{E}{(Z)} = \sum_{j=1}^m\Pr[j\ is\ sampled]\cdot mf_j^2$$
$$= \sum_{j=1}^m \frac{1}{m} \cdot mf_j^2$$
$$= \sum_{j=1}^m f_j^2$$
Calculate $\mathbb{E}(Z^2)$:
$$\mathbb{E}(Z^2)=\sum_{j=1}^m\Pr[j\ is\ sampled]\cdot (mf_j^2)^2$$
$$=\sum_{j=1}^m \frac{1}{m} \cdot (mf_j^2)^2$$
$$=m\sum_{j=1}^m f_j^4$$
Consider two cases:
\begin{itemize}
    \item For each j, $j =\frac{n}{m}$:
    $$\frac{\mathbb{E}(Z^2)}{\mathbb{E}(Z)^2} = \frac{m\sum_{j=1}^m f_j^4}{(\sum_{j=1}^m f_j^2)^2} = \frac{m(\frac{n}{m})^4m}{((\frac{n}{m})^2 m)^2} = 1$$
    This is good, since the bound of the number of samples is $\frac{1}{\epsilon^2}\ln{\frac{1}{\delta}} $.
    
    \item $f_1=n$ and for all j$\in$\{2, ..., m\}, $f_j=0$:
    $$\frac{\mathbb{E}(Z^2)}{\mathbb{E}(Z)^2} = \frac{mn^4}{n^4} = m $$
    This is not good, since the bound of number of samples is $m\frac{1}{\epsilon^2}\ln{\frac{1}{\delta}} $. If m is large, it's going to be large.
\end{itemize}
\newpage
\subsection{Algorithm 1 -- Try 2}
\begin{mdframed}[backgroundcolor=blue!05,topline=false,bottomline=false,leftline=false,rightline=false] 
	\underline{\sc Try 2}
	\begin{itemize}
		\item Sample a j$\in$\{1, 2, ..., m\} $\propto f_j$ i.e. $\frac{f_j}{n}$ (suppose we know $\frac{f_j}{n}$)
		\item Count/Evaluate $f_j$
		\item return $Z=nf_j$
	\end{itemize}
\end{mdframed}
\textbf{Analysis:} Again, we need to evaluate the bound of the number of samples Try2 need.\\
Calculate $\mathbb{E}(Z)$:
$$\mathbb{E}{(Z)} = \sum_{j=1}^m\Pr[j\ is\ sampled]\cdot nf_j$$
$$= \sum_{j=1}^m \frac{f_j}{n}\cdot nf_j$$
$$= \sum_{j=1}^m f_j^2$$
Calculate $\mathbb{E}(Z^2)$:
$$\mathbb{E}(Z^2)=\sum_{j=1}^m\Pr[j\ is\ sampled]\cdot (nf_j)^2$$
$$= \sum_{j=1}^m \frac{f_j}{n} \cdot (nf_j)^2$$
$$=n\sum_{j=1}^m f_j^3$$
Consider four cases:
\begin{itemize}
    \item For each j, $j =\frac{n}{m}$:
    $$\frac{\mathbb{E}(Z^2)}{\mathbb{E}(Z)^2} = \frac{n\sum_{j=1}^m f_j^3}{(\sum_{j=1}^m f_j^2)^2} = \frac{n(\frac{n}{m})^3m}{((\frac{n}{m})^2 m)^2} = 1$$
    This is good, since the bound of the number of samples is $\frac{1}{\epsilon^2}\ln{\frac{1}{\delta}} $.
    
    \item $f_1=n$ and for all j$\in$\{2, ..., m\}, $f_j=0$:
    $$\frac{\mathbb{E}(Z^2)}{\mathbb{E}(Z)^2} = \frac{n\sum_{j=1}^m f_j^3}{(\sum_{j=1}^m f_j^2)^2} = \frac{n(n)^3}{((n)^2)^2} = 1$$
    This is good as well.
    
    \item $f_1=\frac{n}{2}$, for all j$\in$\{2, ..., n/2\}, $f_j=1$ and for others $f_j=0$:
    $$\frac{\mathbb{E}(Z^2)}{\mathbb{E}(Z)^2} = \frac{n\sum_{j=1}^m f_j^3}{(\sum_{j=1}^m f_j^2)^2} = \frac{\frac{n^4}{8}+\frac{n^2}{2}}{(\frac{n^2}{4}+\frac{n}{2})^2} \approx constant$$
    This is good as well.
    
    \item $f_1=\sqrt{n}$, for other j, some $f_j=1$ and some $f_j=0$ s.t. $\sum_{j=1}^m f_j = n$:
    $$\frac{\mathbb{E}(Z^2)}{\mathbb{E}(Z)^2} = \frac{n\sum_{j=1}^m f_j^3}{(\sum_{j=1}^m f_j^2)^2} \approx \frac{n^{\frac{5}{2}}}{n^2} \approx \sqrt{n}$$
    This is not that good, but it's good enough, since the bound of number of samples is approximately $\sqrt{n}\frac{1}{\epsilon^2}\ln{\frac{1}{\delta}} $.
\end{itemize}
Try 2 is an acceptable good approach, but the problem is that we don't know $\frac{f_j}{n}$. So Try 3 will propose an approach that give us $\sqrt{n}$ bounds without the knowledge of $\frac{f_j}{n}$.

\subsection{Algorithm 1 -- Try 3}
\begin{mdframed}[backgroundcolor=blue!05,topline=false,bottomline=false,leftline=false,rightline=false] 
	\underline{\sc Try 3}
	\begin{itemize}
		\item Sample a coordinate r$\in$ A[1...n] u.a.r
		\item j = A[r]
		\item $n_j \equiv$ \# of occurrence of j in A[r...n]
		\item return $Z=(2n_j - 1)n$
	\end{itemize}
\end{mdframed}
\textbf{Analysis:}\\
Calculate $\mathbb{E}(n_j|j)$:\\
Given that we sampled j, we are equally likely to sample any of the $f_j$ occuerence. 
$$\mathbb{E}(n_j|j) = \frac{f_j+(f_j-1)+(f_j-2)+...+1}{f_j}$$
$$=\frac{f_j(f_j+1)}{2f_j}$$
$$=\frac{f_j+1}{2}$$
Calculate $\mathbb{E}(Z|j)$:\\
$$\mathbb{E}(Z|j) = (2\mathbb{E}(n_j|j)-1)n$$
$$=n f_j$$
Calculate $\mathbb{E}(Z)$:\\
$$\mathbb{E}(Z) = \sum_{j=1}^m \Pr[sample\ j]\cdot\mathbb{E}(Z|j)$$
$$=\sum_{j=1}^m \frac{f_j}{n}\cdot nf_j$$
$$=\sum_{j=1}^m f_j^2$$
Calculate $\mathbb{E}(Z^2)$:\\
Given that we sampled j, $Z\leq 2nf_j$.
$$\mathbb{E}(Z^2) = \sum_{j=1}^m \Pr[sample\ j]\cdot\mathbb{E}(Z^2|j)$$
$$\leq \sum_{j=1}^m \frac{f_j}{n}\cdot (2nf_j)^2$$
$$= 4n\sum_{j=1}^m f_j^3$$
$$\leq 4\sqrt{n}(\sum_{j=1}^m f_j^2)^2$$
$$= 4\sqrt{n}\mathbb{E}(Z)^2$$
So $\frac{\mathbb{E}(Z^2)}{\mathbb{E}(Z)^2} \leq 4\sqrt{n}$, this is good, it give us the bound of the number of samples to be $4\sqrt{n}\frac{1}{\epsilon^2}\ln{\frac{1}{\delta}}$ 

\subsection{Algorithm 2}
\begin{mdframed}[backgroundcolor=blue!05,topline=false,bottomline=false,leftline=false,rightline=false] 
	\underline{\sc Algorithm 2}
	\begin{itemize}
		\item C = 0
		\item Sample a g:\ \ [m]$\longrightarrow$ \{1, -1\} from a 4-wise independent hash family. 
		\item When element a arrives:\\ 
		      \centerline{C = C + g(a)}
		\item return $Z=C^2$
	\end{itemize}
\end{mdframed}
\textbf{Analysis:}\\
From the algorithm, we know that $$C = \sum_{a} f_a\cdot g(a)$$
Calculate $\mathbb{E}(Z)$:\\
$$\mathbb{E}(Z) = \mathbb{E}(C^2)$$
$$= \mathbb{E}[\sum_{a=1}^m f_a\cdot g(a))^2]$$
$$= \mathbb{E}[\sum_{a=1}^m f_a^2 + (\sum_{a\neq b}f_af_bg(a)g(b)]$$
$$=F_2 + \sum_{a\neq b}f_af_b \mathbb{E}[g(a)g(b)]$$
Since g is from 4-wise independent hash family, the second term is 0.
$$=F_2$$
Calculate $\mathbb{E}(Z^2)$:\\
$$\mathbb{E}(Z^2) = \mathbb{E}(C^4)$$
$$= \mathbb{E}[\sum_{a=1}^m (f_a\cdot g(a))^4]$$
$$= \mathbb{E}[\sum_{a=1}^m f_a^4 + \sum_{a\neq b}f_a^2f_b^2 + \sum_{a,b,c,d}f_af_bf_cf_dg(a)g(b)g(c)g(d)]$$
$$= \sum_{a=1}^m f_a^4 + \sum_{a\neq b}f_a^2f_b^2 + \sum_{a,b,c,d}f_af_bf_cf_d\mathbb{E}[g(a)g(b)g(c)g(d)]$$
Since g is from 4-wise independent hash family, the third term is 0.
$$=\sum_{a=1}^m f_a^4 + \sum_{a\neq b}f_a^2f_b^2$$
$$=(\sum_{a=1}^m f_j^2)^2$$
$$=F_2^2$$
So $\frac{\mathbb{E}(Z^2)}{\mathbb{E}(Z)^2} = 1$, this is good, since it give us the bound of the number of samples to be $\frac{1}{\epsilon^2}\ln{\frac{1}{\delta}} $

\todo{
Can algorithm 2 be modified to estimate $F_3$?
}

\end{document}

