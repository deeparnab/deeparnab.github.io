\hrule height 1pt
\noindent

\begin{center}
	{\Large \bf Instructions}
\end{center}
\begin{itemize}
	\item {\em Presentation}:
	In most problems below you will be asked to describe an algorithm and analyze it. Please keep in mind the following things
	\begin{itemize}[noitemsep]
		\item 
		Explain the idea behind your algorithm in English. 
		If the idea is clear from the description, you need not even write the pseudocode description. If you do write pseudocode, please provide ``comments'' to explain what your variables, assertions, etc mean.
		
		\item 
		Many times we will ask you to justify your algorithm. You then need to give enough arguments to convince the reader of your algorithm.
		If an argument needs a formal proof, provide it.
		
		\item Always provide the running time of the algorithms you describe. Unless mentioned, this should be in the Big-Oh notation.
	\end{itemize}
	

\item {\em General small print:} Please submit all homework electronically in PDF format ideally typeset using LaTeX. 
%You need to submit only the problems above the line. 
Please try to be concise -- as a rule of thumb do not take more than 1 (LaTeX-ed) page for a solution. 


\item {\em Collaboration Policy:} You are allowed to discuss with other students but are {\em not allowed} to exchange full solutions.
At the beginning of each problem you must write who you discussed with, and what way did the person help you or you help them. This is important. 
If you did not talk with anyone about any of the problems, mention this at the beginning of the homework. You may not consult any solutions on the Internet or from previous years' assignments, whether they are student- or faculty-generated.


\item {\em Extra Credit Problems:} Sometimes I will assign problems for extra-credit. 
Please note the extra-credit problems cannot ``make up'' for losses in the problems for credit.
Often, but not always, they will be harder, and will not be rigorously graded (sometimes it may be a problem we don't 
know the solution to). So {\em try these only after} doing the for-credit problems.

\end{itemize}	

\announce{}
\thispagestyle{empty}
\newpage
\setcounter{page}{1}