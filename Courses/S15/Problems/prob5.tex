\documentclass[11pt]{article}
\usepackage{graphicx}
\usepackage{algorithm,algorithmic}
\usepackage{amsmath, amssymb, amsthm}
\usepackage{url}
\usepackage{fullpage, prettyref}
\usepackage{pstricks,pst-node,pst-text}
\usepackage{boxedminipage}
\usepackage{hyperref}
\usepackage{wrapfig}
\usepackage{ifthen}
\theoremstyle{definition}
\newtheorem{theorem}{Theorem}
\newtheorem{lemma}{Lemma}
\newtheorem{claim}{Claim}
\newtheorem{corollary}{Corollary}
\newtheorem{definition}{Definition}
\newtheorem{proposition}{Proposition}
\newtheorem{fact}{Fact}
\newtheorem{example}{Example}
\newtheorem{exercise}{Exercise}
\newtheorem{assumption}{Assumption}
\newtheorem{observation}{Observation}

\newcommand{\comment}[1]{\textsl{\small[#1]}\marginpar{\tiny\textsc{To Do!}}}
\newcommand{\ignore}[1]{}

\def\eps{\varepsilon}
\def\bar{\overline}
\def\floor#1{\lfloor {#1} \rfloor}
\def\ceil#1{\lceil {#1} \rceil}
\def\script#1{\mathcal{#1}}

\def\plus{{\tt (+)}}
\def\2plus{{\tt (++)}}
\def\3plus{{\tt (+++)}}
\def\4plus{{\tt (++++)}}
\def\5plus{{\tt (+++++)}}

\def\opt{{\tt opt}}
\def\alg{{\tt alg}}

\newenvironment{proofof}[1]{\smallskip\noindent{\bf Proof of #1:}}%
        {\hspace*{\fill}$\Box$\par}

\setlength{\oddsidemargin}{0pt}
\setlength{\evensidemargin}{0pt}
\setlength{\textwidth}{6.5in}
\setlength{\topmargin}{0in}
\setlength{\textheight}{8.5in}
\newlength{\algobox}
\setlength{\algobox}{6.5in}


\begin{document}
\title{{\bf Approximation Algorithms} \\ 
{\normalsize EO249}}
\date{(Problem Set 5) Due: Mar 13th, 2015}
\maketitle
{\small 
Solutions need to be submitted by email to eo249iisc@gmail.com. We prefer latexed solution. 
Before giving the solutions, you should write how many problems you have attempted and how many you think you have solved.
Starred problems are optional and (possibly) more fun.
}
\vspace{1ex}
\def\poly{\mathrm{poly}}
\begin{exercise}
Consider the following LP for the knapsack problem
\begin{align}
\max & \qquad \sum_{i=1}^n p_ix_i & \notag \\
\textrm{such that} &\qquad  \sum_{i=1}^n w_ix_i \leq B\notag \\
& \qquad 0 \leq x_i \leq 1, \qquad \forall i \notag
\end{align}
Here $p_i$ and $w_i$ are the profits and weights of the $i$th item. We may assume $w_i \leq B$ for all $i$.
\begin{itemize}
\item Let $x$ be an optimal basic feasible solution to the above LP. What can you say about the set $\{i: 0 < x_i < 1\}$?
\item Can you use this to argue the integrality gap of the LP $\leq 2$? Can you prove a lower bound on the integrality gap?
\item  Can you use the above to design a PTAS for the knapsack problem, that is an $(1+\eps)$-approximation running in time $n^{f(\eps)}$ for some function $f$?
\end{itemize}
\end{exercise}
\vspace{1ex}

\begin{exercise}
In the survivable network design problem (SNDP), we are given an undirected graph $G$ with costs on edges. We are also given source-sink pairs $(s_i,t_i)$
and an integer $r_i$ associated with each pair.
The goal is to construct a minimum cost subgraph $H$ of $G$ such that for each $i$, there exists at least $r_i$ {\bf edge-disjoint} paths from $s_i$ to $t_i$.
In this exercise we design a $2$-approximation for the problem.
\begin{enumerate}
\item 
Define the following {\bf set function} $r(S) := \max_{i: |S\cap \{s_i,t_i\}|  = 1} r_i$. Prove that $r$ is {\em weakly supermodular}, that is,
for any two subsets $A$ and $B$ one of the two holds:
\[
r(A \cup B)  + r(A \cap B) \geq r(A) + r(B)
\]
\[
r(A\setminus B) + r(B\setminus A) \geq r(A) + r(B)
\]
\item Consider the following LP relaxation for the problem.
\begin{align}
\min & \qquad \sum_{e\in E} c_ex_e & \notag \\
\textrm{such that} &\qquad  x(\delta(S)) \geq r(S) \qquad \forall S\subseteq V\notag \\
& \qquad x_e \geq 0 \notag
\end{align}
where, $\delta(S)$ is the set of edges with exactly one endpoint in the set $S$ and $x(\delta(S))$ is the shorthand for $\sum_{e\in \delta(S)} x_e$.
Prove that the above is a relaxation to the problem. Describe a {\em separation oracle} for the problem, that is, given any $x_e$'s 
describe an efficient algorithm which either proves $x$ is feasible, or demonstrates a set $S$ violating the inequality.

\item 
Prove that for any $x_e$'s, and any subset $A,B \subseteq V$, we have 
\[
x(\delta(A \cup B))  + x(\delta(A \cap B)) \leq x(\delta(A)) + x(\delta(B))
\]
{\bf and}
\[
x(\delta(A\setminus B)) + x(\delta(B\setminus A)) \leq x(\delta(A)) + x(\delta(B))
\]
Note the change in the inequality directions as opposed to (1).
Furthermore, show the following. Given subset $S\subseteq V$, let $\chi_S$ be the row corresponding to the constraint $S$ in the LP above.
Then the above inequalities hold with equality only if  $\chi_A + \chi_B = \chi_{A\cup B} + \chi_{A\cap B}$ for the first inequality and 
$\chi_A + \chi_B = \chi_{A\setminus B} + \chi_{B\setminus A}$.
\item 
\def\cF{\mathcal{F}}
\def\cL{\mathcal{L}}
Let $x$ be a basic feasible solution to the above LP. Call a set $S$ {\bf tight} if and only if $\sum_{e\in \delta(S)} x_e = r(S)$.
Let $\cF$ be the collection of tight sets. Prove that if $S,T\in \cF$, then either $S\cup T,S\cap T$ are in $\cF${\bf and} $\chi_S + \chi_T = \chi_{S\cup T} + \chi_{S\cap T}$ , or 
$S\setminus T, T\setminus S$ are in $\cF$ and $\chi_S +\chi_T = \chi_{S\setminus T} + \chi_{T\setminus S}$.
You may assume (1),(3) for this.

\item
A collection $\cL$ of subsets of $V$ is laminar if for any two sets $S,T \in \cL$, either $S\cap T = \emptyset$, or $S\subseteq T$ or $T\subseteq S$.
Show that there exists a laminar family $\cL$ in $\cF$ such that for any $T\in \cF$, $\chi_T$ can be expressed as a linear combination of 
the rows $\{\chi_S: S\in \cL\}$.
This shows that the rank of the submatrix of tight equalities corresponding to $x$ is at most $|\cL|$.\\
\noindent
{\bf Hint:}  Consider a maximal laminar subcollection of $\cF$, that is, a laminar family $\cL$ such that for every set $S\in \cF\setminus \cL$, there exists $T\in \cL$ with 
$S\cap T \neq \emptyset$ and is a  strict  of $S$ and $T$.

\item 
Use the proof technique from the tree augmentation problem done in class, to prove that some $x_e \geq 1/2$.  Can you see the similarity between laminar families and rooted trees? \\
\noindent
{\bf Hint:} For every edge $e\in E$, put at most one unit of charge on certain sets in $\cL$. In particular, for $e = (u,v)$, consider the minimal set which contains just $u$, just $v$, and both $u$ and $v$.



\end{enumerate}
\end{exercise}

\end{document}