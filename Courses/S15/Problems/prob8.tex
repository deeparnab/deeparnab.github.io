\documentclass[11pt]{article}
\usepackage{graphicx}
\usepackage{algorithm,algorithmic}
\usepackage{amsmath, amssymb, amsthm}
\usepackage{url}
\usepackage{fullpage, prettyref}
\usepackage{pstricks,pst-node,pst-text}
\usepackage{boxedminipage}
\usepackage{hyperref}
\usepackage{wrapfig}
\usepackage{ifthen}
\theoremstyle{definition}
\newtheorem{theorem}{Theorem}
\newtheorem{lemma}{Lemma}
\newtheorem{claim}{Claim}
\newtheorem{corollary}{Corollary}
\newtheorem{definition}{Definition}
\newtheorem{proposition}{Proposition}
\newtheorem{fact}{Fact}
\newtheorem{example}{Example}
\newtheorem{exercise}{Exercise}
\newtheorem{assumption}{Assumption}
\newtheorem{observation}{Observation}

\newcommand{\comment}[1]{\textsl{\small[#1]}\marginpar{\tiny\textsc{To Do!}}}
\newcommand{\ignore}[1]{}

\def\eps{\varepsilon}
\def\bar{\overline}
\def\floor#1{\lfloor {#1} \rfloor}
\def\ceil#1{\lceil {#1} \rceil}
\def\script#1{\mathcal{#1}}

\def\plus{{\tt (+)}}
\def\2plus{{\tt (++)}}
\def\3plus{{\tt (+++)}}
\def\4plus{{\tt (++++)}}
\def\5plus{{\tt (+++++)}}

\def\opt{{\tt opt}}
\def\alg{{\tt alg}}

\newenvironment{proofof}[1]{\smallskip\noindent{\bf Proof of #1:}}%
        {\hspace*{\fill}$\Box$\par}

\setlength{\oddsidemargin}{0pt}
\setlength{\evensidemargin}{0pt}
\setlength{\textwidth}{6.5in}
\setlength{\topmargin}{0in}
\setlength{\textheight}{8.5in}
\newlength{\algobox}
\setlength{\algobox}{6.5in}


\begin{document}
\title{{\bf Approximation Algorithms} \\ 
{\normalsize EO249}}
\date{(Problem Set 8)}
\maketitle
{\small 
Solutions need to be submitted by email to eo249iisc@gmail.com. We prefer latexed solution. 
Before giving the solutions, you should write how many problems you have attempted and how many you think you have solved.
Starred problems are optional and (possibly) more fun.
}
\vspace{1ex}
\def\poly{\mathrm{poly}}
\def\Exp{\mathbf{Exp}}
\def\Pr{\mathbf{Pr}}

\begin{exercise}
Recall the multiway cut problem: given an undirected graph $G=(V,E)$ with costs $c_e$ on edges, and given a set of $k$ vertices $T = \{s_1,\ldots,s_k\}$, find a subset $F$ of edges of minimum cost whose deletion separates every $s_i$ from $s_j$. Consider the following algorithm: let $F_i = \delta(S_i)$ be the minimum cost cut separating $s_i$ from 
$T\setminus s_i$.
\begin{enumerate}
\item Show that $F_i$ can be found in polynomial time.
\item Show that the union of {\bf any} $(k-1)$ of these $F_i$'s forms a valid multiway cut.
\item Prove that the best among these $k$ possibilities of part (2) is of cost $\leq 2(1-1/k) OPT$.
\end{enumerate}
\end{exercise}
\vspace{1ex}

\begin{exercise}
In this exercise we show that the LP relaxation for multicut that we looked in class has integrality gap $\Omega(\log n)$.
To do so, we need a definition. An $(\alpha,d)$-expander is an $n$-vertex graph where each vertex has degree $k$ and for every subset $S\subseteq V$ of $|S|\leq n/2$, we have $|\delta(S)| \geq \alpha d |S|$. We will use the fact that $(\alpha,d)$-expanders exist, where $\alpha$ and $d$ are fixed constants. Let $G$ be such an expander and let $c_e = 1$ for all edges.

\begin{enumerate}
\item For every vertex $u\in V$, let $S_u := \{v: d(u,v) > \frac{\log_d n}{2}\}$. We add an $(s_i,t_i)$ pair with $s_i = u$ and $t_i = v$ for all $v \in S_u$. How many $(s_i,t_i)$ pairs do we add in all?
\item Construct an LP solution to the multicut problem with LP cost  $\leq \frac{nd}{\log_d n}$.
\item In this part, we argue that the optimum multicut must be of cost at least $\Omega(nd\alpha)$. 
\begin{enumerate}
\item Consider the graph from which the optimum multicut $F^*$ has been deleted. 
Argue that the diameter of every connected component is at most $\frac{\log_d n}{2}$.
\item Argue that the size of every connected component is at most $O(\sqrt{n})$.
\item Find a subset of vertices $S$ by taking a union of a bunch of these components such that
$n/3 \leq |S|\leq n/2$. Since $\delta(S) \subseteq F^*$, we get by expander property, $|F^*| \geq \Omega(\alpha d n)$.
\end{enumerate}


\end{enumerate}
\end{exercise}


\end{document}

