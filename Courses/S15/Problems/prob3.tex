\documentclass[11pt]{article}
\usepackage{graphicx}
\usepackage{algorithm,algorithmic}
\usepackage{amsmath, amssymb, amsthm}
\usepackage{url}
\usepackage{fullpage, prettyref}
\usepackage{pstricks,pst-node,pst-text}
\usepackage{boxedminipage}
\usepackage{hyperref}
\usepackage{wrapfig}
\usepackage{ifthen}
\theoremstyle{definition}
\newtheorem{theorem}{Theorem}
\newtheorem{lemma}{Lemma}
\newtheorem{claim}{Claim}
\newtheorem{corollary}{Corollary}
\newtheorem{definition}{Definition}
\newtheorem{proposition}{Proposition}
\newtheorem{fact}{Fact}
\newtheorem{example}{Example}
\newtheorem{exercise}{Exercise}
\newtheorem{assumption}{Assumption}
\newtheorem{observation}{Observation}

\newcommand{\comment}[1]{\textsl{\small[#1]}\marginpar{\tiny\textsc{To Do!}}}
\newcommand{\ignore}[1]{}

\def\eps{\varepsilon}
\def\bar{\overline}
\def\floor#1{\lfloor {#1} \rfloor}
\def\ceil#1{\lceil {#1} \rceil}
\def\script#1{\mathcal{#1}}

\def\plus{{\tt (+)}}
\def\2plus{{\tt (++)}}
\def\3plus{{\tt (+++)}}
\def\4plus{{\tt (++++)}}
\def\5plus{{\tt (+++++)}}

\def\opt{{\tt opt}}
\def\alg{{\tt alg}}

\newenvironment{proofof}[1]{\smallskip\noindent{\bf Proof of #1:}}%
        {\hspace*{\fill}$\Box$\par}

\setlength{\oddsidemargin}{0pt}
\setlength{\evensidemargin}{0pt}
\setlength{\textwidth}{6.5in}
\setlength{\topmargin}{0in}
\setlength{\textheight}{8.5in}
\newlength{\algobox}
\setlength{\algobox}{6.5in}


\begin{document}
\title{{\bf Approximation Algorithms} \\ 
{\normalsize EO249}}
\date{(Problem Set 3) Due: Feb 27th, 2015}
\maketitle
{\small 
Solutions need to be submitted by email to eo249iisc@gmail.com. We prefer latexed solution. 
Before giving the solutions, you should write how many problems you have attempted and how many you think you have solved.
Starred problems are optional and (possibly) more fun.
}
\vspace{1ex}
\def\poly{\mathrm{poly}}

\begin{exercise}
In the {\sc Knapsack} problem, we are given a bound $B$ which is an integer, a set of $n$ items with each item $j$ having an integer  weight $w_j$ and an integer profit $p_j$.
The goal is to choose a subset $S$ of items such that $\sum_{j\in S} w_j \leq B$ and $\sum_{j\in S} p_j$ is maximized.
\begin{enumerate}
\item Consider the greedy algorithm: order the items in decreasing order of $p_j/w_j$ and pick items in this order till you fill the knapsack. 
         What is the approximation factor of this algorithm?
\item Design a polynomial time algorithm for this problem if  $p_j \leq n$ for all $j$. \\
(Hint: Dynamic programming.)
\item Design a polynomial time approximation scheme (PTAS) for the knapsack problem. \\
(Hint: Scale each $p_j$ by a factor such that the new $p'_j$s are all less than $n/\eps$. Then argue that
the optimum value of the scaled instance can't be too much smaller than that of the original instance.)
\end{enumerate}
\end{exercise}
\vspace{1ex}

\begin{exercise}
Describe an {\bf exact} algorithm for TSP on $n$-points running in time $O(2^n\poly(n))$.
\end{exercise}
\vspace{1ex}


\begin{exercise}
Describe a PTAS for Euclidean TSP in $d$-dimensions (we did $d=2$ in class).
What is the dependence on $d$?
\end{exercise}

\end{document}