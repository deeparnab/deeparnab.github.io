\documentclass[11pt]{article}
\usepackage{graphicx}
\usepackage{algorithm,algorithmic}
\usepackage{amsmath, amssymb, amsthm}
\usepackage{url}
\usepackage{fullpage, prettyref}
\usepackage{pstricks,pst-node,pst-text}
\usepackage{boxedminipage}
\usepackage{hyperref}
\usepackage{wrapfig}
\usepackage{ifthen}
\theoremstyle{definition}
\newtheorem{theorem}{Theorem}
\newtheorem{lemma}{Lemma}
\newtheorem{claim}{Claim}
\newtheorem{corollary}{Corollary}
\newtheorem{definition}{Definition}
\newtheorem{proposition}{Proposition}
\newtheorem{fact}{Fact}
\newtheorem{example}{Example}
\newtheorem{exercise}{Exercise}
\newtheorem{assumption}{Assumption}
\newtheorem{observation}{Observation}

\newcommand{\comment}[1]{\textsl{\small[#1]}\marginpar{\tiny\textsc{To Do!}}}
\newcommand{\ignore}[1]{}
\newcommand{\R}{\mathbb{R}}

\def\eps{\varepsilon}
\def\bar{\overline}
\def\floor#1{\lfloor {#1} \rfloor}
\def\ceil#1{\lceil {#1} \rceil}
\def\script#1{\mathcal{#1}}




\def\plus{{\tt (+)}}
\def\2plus{{\tt (++)}}
\def\3plus{{\tt (+++)}}
\def\4plus{{\tt (++++)}}
\def\5plus{{\tt (+++++)}}

\def\opt{{\tt opt}}
\def\alg{{\tt alg}}

\newenvironment{proofof}[1]{\smallskip\noindent{\bf Proof of #1:}}%
        {\hspace*{\fill}$\Box$\par}

\setlength{\oddsidemargin}{0pt}
\setlength{\evensidemargin}{0pt}
\setlength{\textwidth}{6.5in}
\setlength{\topmargin}{0in}
\setlength{\textheight}{8.5in}
\newlength{\algobox}
\setlength{\algobox}{6.5in}


\begin{document}
\title{{\bf Approximation Algorithms: Problem Set 9}}
\date{Due Date: April 18}
\maketitle
{\small 
Solutions need to be submitted by email to \texttt{eo249iisc@gmail.com}. We prefer latexed solution. 
Before giving the solutions, you should write how many problems you have attempted and how many you think you have solved.
Starred problems are optional and (possibly) more fun.
}
\vspace{1ex}
\def\poly{\mathrm{poly}}
\def\Exp{\mathbf{Exp}}
\def\Pr{\mathbf{Pr}}

\begin{exercise}
Let $X$ be a symmetric $n$-by-$n$ real matrix. Prove that the following are equivalent:
\begin{enumerate}
\item[(a)]
$X$ is positive semidefinite.
\item[(b)]
$a^\top X a \geq 0$ for all $a \in \R^n$.
\item[(c)]
{\em Cholesky factorization}: There exists a matrix $U \in \R^{n \times n}$ such that $M = U^\top U$.
\end{enumerate}
\end{exercise}

\vspace{.5cm}
\begin{exercise}
In this exercise, you will work out an algorithm for Cholesky factorization which uses $O(n^3)$ arithmetic operations. Assume the real RAM model of computation, where exact arithmetic operations on real numbers (such as computation of square roots, trig functions, random numbers, etc.) take constant time.

Let $M \in \R^{n \times n}$ be symmetric. The algorithm is recursive. Clearly, when $n=1$, Cholesky factorization is just computing a square root. For $n > 1$, suppose:
$$M = 
\left(
\begin{array}{cc}
\alpha & q^\top\\
q & N
\end{array}
\right)
$$
where $\alpha \in \R$, $q \in \R^{n-1}$ and $N \in \R^{(n-1)\times (n-1)}$. 
\begin{enumerate}
\item[(a)]
Argue that $\alpha \geq 0$ and that $N$ is positive semidefinite.
\item[(b)]
Suppose $\alpha = 0$. Show that $q$ is the zero vector. Assuming the problem can be solved for matrices of size $n-1$, show how to Cholesky factorize $M$. 
\item[(c)]
Suppose $\alpha > 0$. Prove that:
$$
M = 
\left(\begin{array}{cc} \sqrt{\alpha} & 0^\top \\ \frac{1}{\sqrt{\alpha}} q & I_{n-1}\end{array}\right)
\left(\begin{array}{cc} 1 & 0^\top \\ 0 & N-\frac{1}{\alpha}qq^\top\end{array}\right)
\left(\begin{array}{cc} \sqrt{\alpha} & \frac{1}{\sqrt{\alpha}} q^\top \\ 0 & I_{n-1}\end{array}\right)
$$
\item[(d)]
If $\alpha > 0$, show that $N-\frac{1}{\alpha}qq^\top$ is positive semidefinite.
\item[(e)]
If $N-\frac{1}{\alpha}qq^\top$ is recursively factorized as $V^\top V$, show that 
$$U = \left(\begin{array}{cc} \sqrt{\alpha} & \frac{1}{\sqrt{\alpha}} q^\top \\ 0 & V\end{array} \right)$$
satisfies $M = U^\top U$. 
\end{enumerate}
\end{exercise}

\vspace{.5cm}

\begin{exercise}
Recall that in class, we obtained the following SDP relaxation for the \textsf{MAX-2SAT} problem:
\begin{align*}
&\max &\sum_{0 \leq i \leq j \leq n} (a_{i,j} (1+u_i \cdot u_j) +b_{i,j} (1-u_i \cdot u_j))\\
&\text{subject to} & u_i \cdot u_i =1 && 0 \leq i \leq n\\
& & u_i \in \R^{n+1} && 0 \leq i \leq n
\end{align*}
Use the Goemans-Williamson rounding algorithm to show that $\mathbb{E}[\mathsf{Alg}] \geq 0.878 \mathsf{Opt}$, where $\mathsf{Alg}$ is the value of the solution generated by the rounding algorithm.
\end{exercise}

\vspace{.5cm}
\begin{exercise}
\begin{enumerate}
\item[(a)]
Let $p(x)$ be a univariate polynomial of degree $d$ with real coefficients. We say $p$ is a {\em sum of squares} if $p(x) = q_1(x)^2 + q_2(x)^2 + \cdot q_m(x)^2$ for some polynomials $q_1, ..., q_m$ (univariate with real coefficients). Formulate the problem of deciding whether $p$ is a sum of squares as a SDP.
\item[(b)]
A polynomial $p(x)$ is {\em non-negative} if $p(x) \geq 0$ for all $x \in \R$. Show that $p$ is non-negative if and only if it is a sum of squares. \textbf{Hint}: First prove when $p$ is a quadratic. Handle the general case by factoring $p$ into quadratics.
\item[(c)]
Given $p(x)$, formulate the problem of finding its global minimum $\min\{p(t) \mid t \in \R\}$ as the solution of a semidefinite program (by using parts (a) and (b)).
\item[(d$^\ast$)] Extend part (a) to multivariate polynomials.
\end{enumerate}
\vspace{.3cm}
\textbf{Note:} Part (b) of this problem doesn't extend to multivariate polynomials. For example, it can be verified that the {\em Motzkin polynomial} $p(x,y) = 1+x^2y^2 (x^2 + y^2 - 3)$ is non-negative but is not a sum of squares. However, for many families of multivariate polynomials, this approach is a powerful algorithm for finding the global minimum.
\end{exercise}


\end{document}

