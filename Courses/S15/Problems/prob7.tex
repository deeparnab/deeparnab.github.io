\documentclass[11pt]{article}
\usepackage{graphicx}
\usepackage{algorithm,algorithmic}
\usepackage{amsmath, amssymb, amsthm}
\usepackage{url}
\usepackage{fullpage, prettyref}
\usepackage{pstricks,pst-node,pst-text}
\usepackage{boxedminipage}
\usepackage{hyperref}
\usepackage{wrapfig}
\usepackage{ifthen}
\theoremstyle{definition}
\newtheorem{theorem}{Theorem}
\newtheorem{lemma}{Lemma}
\newtheorem{claim}{Claim}
\newtheorem{corollary}{Corollary}
\newtheorem{definition}{Definition}
\newtheorem{proposition}{Proposition}
\newtheorem{fact}{Fact}
\newtheorem{example}{Example}
\newtheorem{exercise}{Exercise}
\newtheorem{assumption}{Assumption}
\newtheorem{observation}{Observation}

\newcommand{\comment}[1]{\textsl{\small[#1]}\marginpar{\tiny\textsc{To Do!}}}
\newcommand{\ignore}[1]{}

\def\eps{\varepsilon}
\def\bar{\overline}
\def\floor#1{\lfloor {#1} \rfloor}
\def\ceil#1{\lceil {#1} \rceil}
\def\script#1{\mathcal{#1}}

\def\plus{{\tt (+)}}
\def\2plus{{\tt (++)}}
\def\3plus{{\tt (+++)}}
\def\4plus{{\tt (++++)}}
\def\5plus{{\tt (+++++)}}

\def\opt{{\tt opt}}
\def\alg{{\tt alg}}

\newenvironment{proofof}[1]{\smallskip\noindent{\bf Proof of #1:}}%
        {\hspace*{\fill}$\Box$\par}

\setlength{\oddsidemargin}{0pt}
\setlength{\evensidemargin}{0pt}
\setlength{\textwidth}{6.5in}
\setlength{\topmargin}{0in}
\setlength{\textheight}{8.5in}
\newlength{\algobox}
\setlength{\algobox}{6.5in}


\begin{document}
\title{{\bf Approximation Algorithms} \\ 
{\normalsize EO249}}
\date{(Problem Set 7)}
\maketitle
{\small 
Solutions need to be submitted by email to eo249iisc@gmail.com. We prefer latexed solution. 
Before giving the solutions, you should write how many problems you have attempted and how many you think you have solved.
Starred problems are optional and (possibly) more fun.
}
\vspace{1ex}
\def\poly{\mathrm{poly}}
\def\Exp{\mathbf{Exp}}
\def\Pr{\mathbf{Pr}}

\begin{exercise}
Consider the following linear programming relaxation for the minimum spanning tree problem.
Let the given graph be $G = (V,E)$. Given a subset $S$ of the vertices, let $E[S]$ denote the set of edges with {\bf both} endpoints in $S$.
\begin{align*}
\min & \quad \sum_{e\in E} c_ex_e & \\
\text{such that} & \quad \sum_{e\in E[S]} x_e \leq |S| - 1 & \forall S\subseteq V \\
& \quad \sum_{e\in E} x_e = |V| - 1
\end{align*}
\begin{enumerate}
\item Write the dual to the above LP. 
{\bf Caution:} Note that it is a minimization problem and the first inequality is a $\leq$.
\item (*) Recall the greedy algorithm for the minimum spanning tree problem: till one gets a spanning tree traverse edges in increasing cost order and select it only if it doesn't form a cycle with the existing selected edges. Prove that this is the optimum solution by giving a {\bf dual fitting}. That is, find a feasible dual solution which equals the cost of the tree returned by the above algorithm.
\end{enumerate}
\end{exercise}
\vspace{1ex}

\begin{exercise}
In class, we did the online set cover problem where every set had cost $= 1$. Consider the general cost case where set $S_i$ has cost $c_i$. Modify the online algorithm done in class. You can just do the online fractional set cover problem.

{\bf Hint:} In each run of the while loop, we maintained that the total increase in the primal cost is at most $1$. How will you modify $x_i$ to maintain the same invariant? How will you initialize the $x_i$'s?

\end{exercise}


\end{document}

