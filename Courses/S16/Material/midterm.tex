\documentclass[11pt]{article}
\usepackage{graphicx}
\usepackage{algorithm,algorithmic}
\usepackage{amsmath, amssymb, amsthm}
\usepackage{url}
\usepackage{fullpage, prettyref}
\usepackage{pstricks}
\usepackage{boxedminipage}
\usepackage{hyperref}
\usepackage{wrapfig}
\usepackage{ifthen}
\newtheorem{theorem}{Theorem}
\newtheorem{lemma}{Lemma}
\newtheorem{claim}{Claim}
\newtheorem{corollary}{Corollary}
\newtheorem{definition}{Definition}
\newtheorem{proposition}{Proposition}
\newtheorem{fact}{Fact}
\newtheorem{example}{Example}
\newtheorem{exercise}{Exercise}
\newtheorem{assumption}{Assumption}
\newtheorem{observation}{Observation}

\newcommand{\comment}[1]{\textsl{\small[#1]}\marginpar{\tiny\textsc{To Do!}}}
\newcommand{\ignore}[1]{}

\def\eps{\varepsilon}
\def\bar{\overline}
\def\floor#1{\lfloor {#1} \rfloor}
\def\ceil#1{\lceil {#1} \rceil}
\def\script#1{\mathcal{#1}}

\def\plus{{\tt (+)}}
\def\2plus{{\tt (++)}}
\def\3plus{{\tt (+++)}}
\def\4plus{{\tt (++++)}}
\def\5plus{{\tt (+++++)}}

\def\opt{{\tt opt}}
\def\alg{{\tt alg}}

\newenvironment{proofof}[1]{\smallskip\noindent{\bf Proof of #1:}}%
        {\hspace*{\fill}$\Box$\par}

\setlength{\oddsidemargin}{0pt}
\setlength{\evensidemargin}{0pt}
\setlength{\textwidth}{6.5in}
\setlength{\topmargin}{0in}
\setlength{\textheight}{8.5in}
\newlength{\algobox}
\setlength{\algobox}{6.5in}

\newcommand{\cA}{{\cal A}}
\newcommand{\cB}{\mathcal{B}}
\newcommand{\cC}{{\cal C}}
\newcommand{\cD}{\mathcal{D}}
\newcommand{\cE}{{\cal E}}
\newcommand{\cF}{\mathcal{F}}
\newcommand{\cG}{\mathcal{G}}
\newcommand{\cH}{{\cal H}}
\newcommand{\cI}{{\mathcal{I}}}
\newcommand{\cJ}{{\cal J}}
\newcommand{\cL}{{\cal L}}
\newcommand{\cM}{{\cal M}}
\newcommand{\cP}{\mathcal{P}}
\newcommand{\cQ}{\mathcal{Q}}
\newcommand{\cR}{{\cal R}}
\newcommand{\cS}{\mathcal{S}}
\newcommand{\cT}{{\cal T}}
\newcommand{\cU}{{\cal U}}
\newcommand{\cV}{{\cal V}}
\newcommand{\cX}{{\cal X}}


\newcommand{\R}{\mathbb R}
\newcommand{\N}{\mathbb N}
\newcommand{\F}{\mathbb F}
\newcommand{\Z}{{\mathbb Z}}
\renewcommand{\eps}{\varepsilon}
\newcommand{\lam}{\lambda}
\newcommand{\sgn}{\mathrm{sgn}}
\newcommand{\poly}{\mathrm{poly}}
\newcommand{\polylog}{\mathrm{polylog}}
\newcommand{\littlesum}{\mathop{{\textstyle \sum}}}
\newcommand{\half}{{\textstyle \frac12}}
\newcommand{\la}{\langle}
\newcommand{\ra}{\rangle}
\newcommand{\wh}{\widehat}
\newcommand{\wt}{\widetilde}
\newcommand{\calE}{{\cal E}}
\newcommand{\calL}{{\cal L}}
\newcommand{\calF}{{\cal F}}
\newcommand{\calW}{{\cal W}}
\newcommand{\calH}{{\cal H}}
\newcommand{\calN}{{\cal N}}
\newcommand{\calO}{{\cal O}}
\newcommand{\calP}{{\cal P}}
\newcommand{\calV}{{\cal V}}
\newcommand{\calS}{{\cal S}}
\newcommand{\calT}{{\cal T}}
\newcommand{\calD}{{\cal D}}
\newcommand{\calC}{{\cal C}}
\newcommand{\calX}{{\cal X}}
\newcommand{\calY}{{\cal Y}}
\newcommand{\calZ}{{\cal Z}}
\newcommand{\calA}{{\cal A}}
\newcommand{\calB}{{\cal B}}
\newcommand{\calG}{{\cal G}}
\newcommand{\calI}{{\cal I}}
\newcommand{\calJ}{{\cal J}}
\newcommand{\calR}{{\cal R}}
\newcommand{\calK}{{\cal K}}
\newcommand{\calU}{{\cal U}}
\newcommand{\barx}{\overline{x}}
\newcommand{\bary}{\overline{y}}

\newcommand{\ba}{\boldsymbol{a}}
\newcommand{\bb}{\boldsymbol{b}}
\newcommand{\bp}{\boldsymbol{p}}
\newcommand{\bt}{\boldsymbol{t}}
\newcommand{\bv}{\boldsymbol{v}}
\newcommand{\bx}{\boldsymbol{x}}
\newcommand{\by}{\boldsymbol{y}}
\newcommand{\bz}{\boldsymbol{z}}
\newcommand{\br}{\boldsymbol{r}}
\newcommand{\bh}{\boldsymbol{h}}

\newcommand{\bA}{\boldsymbol{A}}
\newcommand{\bD}{\boldsymbol{D}}
\newcommand{\bG}{\boldsymbol{G}}

\newcommand{\bR}{\boldsymbol{R}}
\newcommand{\bS}{\boldsymbol{S}}
\newcommand{\bX}{\boldsymbol{X}}
\newcommand{\bY}{\boldsymbol{Y}}
\newcommand{\bZ}{\boldsymbol{Z}}

\newcommand{\NN}{\mathbb{N}}
\newcommand{\RR}{\mathbb{R}}

\newcommand{\abs}[1]{\left\lvert #1 \right\rvert}
\newcommand{\norm}[1]{\left\lVert #1 \right\rVert}
\renewcommand{\ceil}[1]{\lceil#1\rceil}
\newcommand{\Exp}{\EX}
\renewcommand{\floor}[1]{\lfloor#1\rfloor}

\newcommand{\EX}{\hbox{\bf E}}
\newcommand{\prob}{{\rm Prob}}

\newcommand{\gset}{Y}
\newcommand{\gcol}{{\cal Y}}

%% Hyper-linked References
\newcommand{\Sec}[1]{\hyperref[sec:#1]{\S\ref*{sec:#1}}} %section
\newcommand{\Eqn}[1]{\hyperref[eq:#1]{(\ref*{eq:#1})}} %equation
\newcommand{\Fig}[1]{\hyperref[fig:#1]{Fig.\,\ref*{fig:#1}}} %figure
\newcommand{\Tab}[1]{\hyperref[tab:#1]{Tab.\,\ref*{tab:#1}}} %table
\newcommand{\Thm}[1]{\hyperref[thm:#1]{Theorem\,\ref*{thm:#1}}} %theorem
\newcommand{\Fact}[1]{\hyperref[fact:#1]{Fact\,\ref*{fact:#1}}} %fact
\newcommand{\Lem}[1]{\hyperref[lem:#1]{Lemma\,\ref*{lem:#1}}} %lemma
\newcommand{\Prop}[1]{\hyperref[prop:#1]{Prop.~\ref*{prop:#1}}} %property
\newcommand{\Cor}[1]{\hyperref[cor:#1]{Corollary~\ref*{cor:#1}}} %corollary
\newcommand{\Conj}[1]{\hyperref[conj:#1]{Conjecture~\ref*{conj:#1}}} %conjecture
\newcommand{\Def}[1]{\hyperref[def:#1]{Definition~\ref*{def:#1}}} %definition
\newcommand{\Alg}[1]{\hyperref[alg:#1]{Alg.~\ref*{alg:#1}}} %algorithm
\newcommand{\Ex}[1]{\hyperref[ex:#1]{Ex.~\ref*{ex:#1}}} %example
\newcommand{\Clm}[1]{\hyperref[clm:#1]{Claim~\ref*{clm:#1}}} %example

\newcommand{\Sol}{{\bf Solution sketch:} }

\usepackage{MnSymbol,wasysym}

\newcommand{\el}{\ensuremath{\ell}}


\usepackage{times}
\usepackage{graphicx}
\usepackage{algorithm,algorithmic}
\usepackage{amsmath, amssymb, amsthm}
\usepackage{url}
\usepackage{fullpage, prettyref}
\usepackage{pstricks,pst-node}
\usepackage{boxedminipage}
\usepackage{hyperref}
\usepackage{wrapfig}
\usepackage{ifthen}

\def\eps{\varepsilon}
\def\bar{\overline}
\def\floor#1{\lfloor {#1} \rfloor}
\def\ceil#1{\lceil {#1} \rceil}
\def\script#1{\mathcal{#1}}

\def\opt{{\tt opt}}
\def\alg{{\tt alg}}

\def\Pr{\mathbf{Pr}}
\def\Exp{\mathbf{Exp}}
\def\Var{\mathbf{Var}}
 
\setlength{\oddsidemargin}{0pt}
\setlength{\evensidemargin}{0pt}
\setlength{\textwidth}{6.5in}
\setlength{\topmargin}{0in}
\setlength{\textheight}{8.5in}
\setlength{\algobox}{6.5in}
\newcommand{\supp}{{\tt supp}}
\newcommand{\rank}{{\tt rank}}
\def\Pr{\mathbf{Pr}}
\def\Exp{\mathbf{Exp}}
\def\Var{\mathbf{Var}}

\begin{document}

\title{E0234 Randomized Algorithms}
\author{\bf Midterm}
\date{29th Feb, 2016. 1:30pm to 4:30pm.}
\maketitle
\thispagestyle{empty}
\def\poly{{\tt poly}}
%You are allowed one A4 sheet of paper in your own handwriting which needs to be submitted with your answer sheet. 
\begin{center}
{\small 
Good luck!
}


\end{center}
%\hline

\begin{enumerate}
	

\item 
\begin{enumerate}
\item[(a)] Prove: for any two random variables $X,Y$, $\Exp[X+Y] = \Exp[X] + \Exp[Y]$.
\item[(b)]
Prove or disprove: for independent random variables $X,Y$, \mbox{$\Var[XY] = \Var[X]\cdot \Var[Y]$.}
%\item[(b)] Prove or disprove: for any two random variables $X$ and $Y$ (not necessarily independent), \\ $\Exp[XY] \leq \sqrt{\Exp[X^2]}\sqrt{\Exp[Y^2]}$.
%\item[(b)]Let $X$ and $Y$ be random variables (not necessarily independent), and let $E(X,Y)$ be an event. Suppose $\Pr_{X,Y}[E(X,Y)] \geq \beta$. Then, show that $\Pr_X[\Pr_Y[E(X,Y)] \geq \beta/2] \geq \beta/2$.  

\end{enumerate}
\vspace{0.5ex}

\item 
Consider a set system where each set has exactly $10$ elements and every element is present in exactly $10$ sets. 
Can you colour the elements red or blue such that every set has elements of both colours?
%
%\item A hypergraph is $k$-uniform if each hyperedge has exactly $k$ vertices, and is called
%$k$-regular if every vertex is in exactly $k$ different hyperedges. A hypergraph is $2$-colorable if every vertex can be coloured either red or blue such that there are no monochromatic hyperedges. Prove that for $k>10$, every $k$-uniform, $k$-regular hypergraph is $2$-colourable.
\vspace{0.5ex}

\item Consider the random graph $G_{n,p}$ when $p = cn^{-2/3}$, and let $X$ be the number of $4$-cliques in the graph. 
\begin{enumerate}
	\item  What is $\Exp[X]$?
	\item  What is an upper bound on $\Var[X]$?
	\item  What is $\lim_{n\to \infty} \Pr[X = 0]$? 
\end{enumerate}
\vspace{0.5ex}

%\item Prove that for any two random variables $X$ and $Y$ (not necessarily independent), \\ $\Exp[XY] \leq \sqrt{\Exp[X^2]}\sqrt{\Exp[Y^2]}$.
%Assume $X$ and $Y$ are discrete.
%\item 
%$X$ is a random variable with expectation $\mu > 0$ and standard deviation $\sigma$. 
%Prove $\Pr[X > 0] \geq \frac{\mu^2}{\sigma^2 + \mu^2}$. You are allowed to use the statement from the previous question without proof.
%\vspace{0.5ex}
%\item Define  $S_n := |X_1 + \cdots + X_n|$ where each $X_i$'s are independent random variables which are  $\pm 1$ with probability $1/2$.
%Prove $\Exp[S_n] = \Theta(\sqrt{n})$.
%\vspace{0.5ex}
\item
Suppose you can draw independent samples of a real random variable $X$ that has expectation $0$ and standard deviation $\sigma$. Explain how to use only $O(\log n)$ samples from this source to generate a random variable $Y$ with expectation $\mu$ such that $\Pr[|Y-\mu| > 2 \sigma] < 1/n$. 
\vspace{0.5ex}
\item 
Consider the following algorithm for the independent set problem on an $n$-node graph. Sample a random permutation $\sigma$ of $\{1,2,\ldots,n\}$. Initialize $I$ to $\emptyset$.
For $i=1$ to $n$, place $\sigma(i)$ in $I$ if it doesn't have an edge to any vertex in $I$.  In expectation, how large an independent set do you pick?
\vspace{0.5ex}




\item  In $d$-dimensions, there can be at most $d$ unit vectors which are orthogonal to each other.
Call a pair of unit vectors $\eps$-orthogonal if $|u^\top v|\leq \eps$. How large (in cardinality) a set of pairwise $\eps$-orthogonal unit vectors can you construct in $d$-dimensions?


\end{enumerate}

\end{document}