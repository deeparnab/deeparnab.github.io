\documentclass[11pt]{article}
\usepackage{graphicx}
\usepackage{algorithm,algorithmic}
\usepackage{amsmath, amssymb, amsthm}
\usepackage{url}
\usepackage{fullpage, prettyref}
\usepackage{pstricks}
\usepackage{boxedminipage}
\usepackage{hyperref}
\usepackage{wrapfig}
\usepackage{ifthen}
\newtheorem{theorem}{Theorem}
\newtheorem{lemma}{Lemma}
\newtheorem{claim}{Claim}
\newtheorem{corollary}{Corollary}
\newtheorem{definition}{Definition}
\newtheorem{proposition}{Proposition}
\newtheorem{fact}{Fact}
\newtheorem{example}{Example}
\newtheorem{exercise}{Exercise}
\newtheorem{assumption}{Assumption}
\newtheorem{observation}{Observation}

\newcommand{\comment}[1]{\textsl{\small[#1]}\marginpar{\tiny\textsc{To Do!}}}
\newcommand{\ignore}[1]{}

\def\eps{\varepsilon}
\def\bar{\overline}
\def\floor#1{\lfloor {#1} \rfloor}
\def\ceil#1{\lceil {#1} \rceil}
\def\script#1{\mathcal{#1}}

\def\plus{{\tt (+)}}
\def\2plus{{\tt (++)}}
\def\3plus{{\tt (+++)}}
\def\4plus{{\tt (++++)}}
\def\5plus{{\tt (+++++)}}

\def\opt{{\tt opt}}
\def\alg{{\tt alg}}

\newenvironment{proofof}[1]{\smallskip\noindent{\bf Proof of #1:}}%
        {\hspace*{\fill}$\Box$\par}

\setlength{\oddsidemargin}{0pt}
\setlength{\evensidemargin}{0pt}
\setlength{\textwidth}{6.5in}
\setlength{\topmargin}{0in}
\setlength{\textheight}{8.5in}
\newlength{\algobox}
\setlength{\algobox}{6.5in}

\newcommand{\cA}{{\cal A}}
\newcommand{\cB}{\mathcal{B}}
\newcommand{\cC}{{\cal C}}
\newcommand{\cD}{\mathcal{D}}
\newcommand{\cE}{{\cal E}}
\newcommand{\cF}{\mathcal{F}}
\newcommand{\cG}{\mathcal{G}}
\newcommand{\cH}{{\cal H}}
\newcommand{\cI}{{\mathcal{I}}}
\newcommand{\cJ}{{\cal J}}
\newcommand{\cL}{{\cal L}}
\newcommand{\cM}{{\cal M}}
\newcommand{\cP}{\mathcal{P}}
\newcommand{\cQ}{\mathcal{Q}}
\newcommand{\cR}{{\cal R}}
\newcommand{\cS}{\mathcal{S}}
\newcommand{\cT}{{\cal T}}
\newcommand{\cU}{{\cal U}}
\newcommand{\cV}{{\cal V}}
\newcommand{\cX}{{\cal X}}


\newcommand{\R}{\mathbb R}
\newcommand{\N}{\mathbb N}
\newcommand{\F}{\mathbb F}
\newcommand{\Z}{{\mathbb Z}}
\renewcommand{\eps}{\varepsilon}
\newcommand{\lam}{\lambda}
\newcommand{\sgn}{\mathrm{sgn}}
\newcommand{\poly}{\mathrm{poly}}
\newcommand{\polylog}{\mathrm{polylog}}
\newcommand{\littlesum}{\mathop{{\textstyle \sum}}}
\newcommand{\half}{{\textstyle \frac12}}
\newcommand{\la}{\langle}
\newcommand{\ra}{\rangle}
\newcommand{\wh}{\widehat}
\newcommand{\wt}{\widetilde}
\newcommand{\calE}{{\cal E}}
\newcommand{\calL}{{\cal L}}
\newcommand{\calF}{{\cal F}}
\newcommand{\calW}{{\cal W}}
\newcommand{\calH}{{\cal H}}
\newcommand{\calN}{{\cal N}}
\newcommand{\calO}{{\cal O}}
\newcommand{\calP}{{\cal P}}
\newcommand{\calV}{{\cal V}}
\newcommand{\calS}{{\cal S}}
\newcommand{\calT}{{\cal T}}
\newcommand{\calD}{{\cal D}}
\newcommand{\calC}{{\cal C}}
\newcommand{\calX}{{\cal X}}
\newcommand{\calY}{{\cal Y}}
\newcommand{\calZ}{{\cal Z}}
\newcommand{\calA}{{\cal A}}
\newcommand{\calB}{{\cal B}}
\newcommand{\calG}{{\cal G}}
\newcommand{\calI}{{\cal I}}
\newcommand{\calJ}{{\cal J}}
\newcommand{\calR}{{\cal R}}
\newcommand{\calK}{{\cal K}}
\newcommand{\calU}{{\cal U}}
\newcommand{\barx}{\overline{x}}
\newcommand{\bary}{\overline{y}}

\newcommand{\ba}{\boldsymbol{a}}
\newcommand{\bb}{\boldsymbol{b}}
\newcommand{\bp}{\boldsymbol{p}}
\newcommand{\bt}{\boldsymbol{t}}
\newcommand{\bv}{\boldsymbol{v}}
\newcommand{\bx}{\boldsymbol{x}}
\newcommand{\by}{\boldsymbol{y}}
\newcommand{\bz}{\boldsymbol{z}}
\newcommand{\br}{\boldsymbol{r}}
\newcommand{\bh}{\boldsymbol{h}}

\newcommand{\bA}{\boldsymbol{A}}
\newcommand{\bD}{\boldsymbol{D}}
\newcommand{\bG}{\boldsymbol{G}}

\newcommand{\bR}{\boldsymbol{R}}
\newcommand{\bS}{\boldsymbol{S}}
\newcommand{\bX}{\boldsymbol{X}}
\newcommand{\bY}{\boldsymbol{Y}}
\newcommand{\bZ}{\boldsymbol{Z}}

\newcommand{\NN}{\mathbb{N}}
\newcommand{\RR}{\mathbb{R}}

\newcommand{\abs}[1]{\left\lvert #1 \right\rvert}
\newcommand{\norm}[1]{\left\lVert #1 \right\rVert}
\renewcommand{\ceil}[1]{\lceil#1\rceil}
\newcommand{\Exp}{\EX}
\renewcommand{\floor}[1]{\lfloor#1\rfloor}

\newcommand{\EX}{\hbox{\bf E}}
\newcommand{\prob}{{\rm Prob}}

\newcommand{\gset}{Y}
\newcommand{\gcol}{{\cal Y}}

%% Hyper-linked References
\newcommand{\Sec}[1]{\hyperref[sec:#1]{\S\ref*{sec:#1}}} %section
\newcommand{\Eqn}[1]{\hyperref[eq:#1]{(\ref*{eq:#1})}} %equation
\newcommand{\Fig}[1]{\hyperref[fig:#1]{Fig.\,\ref*{fig:#1}}} %figure
\newcommand{\Tab}[1]{\hyperref[tab:#1]{Tab.\,\ref*{tab:#1}}} %table
\newcommand{\Thm}[1]{\hyperref[thm:#1]{Theorem\,\ref*{thm:#1}}} %theorem
\newcommand{\Fact}[1]{\hyperref[fact:#1]{Fact\,\ref*{fact:#1}}} %fact
\newcommand{\Lem}[1]{\hyperref[lem:#1]{Lemma\,\ref*{lem:#1}}} %lemma
\newcommand{\Prop}[1]{\hyperref[prop:#1]{Prop.~\ref*{prop:#1}}} %property
\newcommand{\Cor}[1]{\hyperref[cor:#1]{Corollary~\ref*{cor:#1}}} %corollary
\newcommand{\Conj}[1]{\hyperref[conj:#1]{Conjecture~\ref*{conj:#1}}} %conjecture
\newcommand{\Def}[1]{\hyperref[def:#1]{Definition~\ref*{def:#1}}} %definition
\newcommand{\Alg}[1]{\hyperref[alg:#1]{Alg.~\ref*{alg:#1}}} %algorithm
\newcommand{\Ex}[1]{\hyperref[ex:#1]{Ex.~\ref*{ex:#1}}} %example
\newcommand{\Clm}[1]{\hyperref[clm:#1]{Claim~\ref*{clm:#1}}} %example

\newcommand{\Sol}{{\bf Solution sketch:} }

\usepackage{MnSymbol,wasysym}

\newcommand{\el}{\ensuremath{\ell}}


\usepackage{times}
\usepackage{graphicx}
\usepackage{algorithm,algorithmic}
\usepackage{amsmath, amssymb, amsthm}
\usepackage{url}
\usepackage{fullpage, prettyref}
\usepackage{pstricks,pst-node}
\usepackage{boxedminipage}
\usepackage{hyperref}
\usepackage{wrapfig}
\usepackage{ifthen}

\def\eps{\varepsilon}
\def\bar{\overline}
\def\floor#1{\lfloor {#1} \rfloor}
\def\ceil#1{\lceil {#1} \rceil}
\def\script#1{\mathcal{#1}}

\def\opt{{\tt opt}}
\def\alg{{\tt alg}}

\def\Pr{\mathbf{Pr}}
\def\Exp{\mathbf{Exp}}
\def\Var{\mathbf{Var}}
 
\setlength{\oddsidemargin}{0pt}
\setlength{\evensidemargin}{0pt}
\setlength{\textwidth}{6.5in}
\setlength{\topmargin}{0in}
\setlength{\textheight}{8.5in}
\setlength{\algobox}{6.5in}
\newcommand{\supp}{{\tt supp}}
\newcommand{\rank}{{\tt rank}}
\def\Pr{\mathbf{Pr}}
\def\Exp{\mathbf{Exp}}
\def\Var{\mathbf{Var}}

\begin{document}

\title{E0234 Randomized Algorithms}
\author{\bf Endterm}
\date{27th Apr, 2016}
\maketitle
\thispagestyle{empty}
\def\poly{{\tt poly}}
%You are allowed one A4 sheet of paper in your own handwriting which needs to be submitted with your answer sheet. 
\begin{center}
{\small 
Good luck!
}


\end{center}
%\hline

\begin{enumerate}
\item ({\bf 10 points.})
Suppose we get $k$ i.i.d. samples  $X_1,\ldots,X_k$ from a distribution with mean $\mu$ and std dev $\sigma$, both of which are {\em unknown}.
We have seen that $Z := \frac{1}{k}\sum_{i=1}^k X_i$ is an unbiased estimator of $\mu$. Describe an unbiased estimator of the variance, $\sigma^2$.

\vspace{.5in}

\item ({\bf 10 points}) $A$ is an array of $n$ distinct numbers. Consider the following algorithm to estimate the approximate median of $A$: sample $k$ entries of $A$ uniformly at random to get the set $R$ and return $r$, the median of $R$. We wish to have the following guarantee:
\[
\Pr\left[\mathsf{rank}(r) \in \left[\frac{n}{2} - \eps n , \frac{n}{2} + \eps n \right] \right] \geq 1-\delta
\]
where $\mathsf{rank}(r)$ is the index of $r$ in the sorted array $A$. How big does $k$ need to be in terms $n,\eps$ and $\delta$?

\vspace{.5in}

\item ({\bf 15 points})
\begin{itemize}
\item[(a)] ({\bf 5 points})
Let $X_0 = 0$, and for $j \geq 0$, let $X_{j+1}$ be chosen uniformly from the real interval $[X_j, 1]$. Show that the sequence $Y_k = 2^k (1-X_k)$ is a martingale.

\item[(b)] ({\bf 10 points})
Suppose $G = G(n,dn)$ is a random graph with $n$ vertices and $dn$ edges chosen uniformly at random among all possible edges. Let $MC_n$ denote the size of the maximum cut in $G$. Prove that with constant probability, $$1-\frac{4}{\sqrt{dn}}< \frac{MC_n}{\EX[MC_n]} < 1+\frac{4}{\sqrt{dn}}$$
\end{itemize}

\vspace{.5in}

	
\item ({\bf 10+5 points})
Let $P$ be an undirected path starting from vertex $1$ on the left to vertex $n$ in the right. Given a permutation $\pi$ of $\{1,2,\ldots,n\}$, orient the edges of $P$
as follows: $(i,i+1)$ is oriented from left to right if $\pi(i) < \pi(i+1)$, and right to left otherwise. Let $Z_\pi$ denote the length of the longest left-to-right directed path in this orientation of $P$.
\begin{enumerate}
	\item Find the largest $k$ such that $\Pr[Z_\pi \geq k] \geq 1/2$, where the probability is over a uniformly chosen random permutation $\pi$. We are interested in the asymptotic relationship between $k$ and $n$, if any.
	\item {\bf Bonus 5 points.} How will you use the above fact to design an algorithm to find long paths in Hamiltonian graphs?
\end{enumerate}


\vspace{.5in}


\item ({\bf 15 points})
Let $G$ be an undirected, non-bipartite and  connected graph with $n$ vertices. Consider two independent random walks starting at two nodes $u$ and $v$ respectively. Show that the expected number of steps for the two walks to meet is $O(n^6)$.

\vspace{.5in}


\item ({\bf 20 points.})
Let $A$ be an $n\times n$ matrix with $|A_{ij}| \leq 1$ entries. In class, we showed the existence of $x\in \{\pm 1\}^n$ such that $||Ax||_\infty \leq \sqrt{2n\ln(2n)}$.
In this problem we see a better result in case $A$ is row and column sparse. Suppose $A$ has at most $k$ non-zero entries in each row and column.
\begin{enumerate}
	\item \textbf{(5 points)} Let $x$ be a random $\{\pm 1\}^n$ vector. For any fixed row $i$, upper bound the probability that $|a_i^\top x| \geq \beta$ where $a_i$ is the $i$th row of $A$.
	\item \textbf{(5 points)} Prove there exists a $\pm 1$ vector $x$ with $||Ax||_\infty = O(\sqrt{k\ln n})$.
	\item \textbf{(10 points)} Prove there exists a $\pm 1$ vector $x$ with $||Ax||_\infty = O(\sqrt{k\ln k})$. \textbf{Hint}: LLL.
\end{enumerate}





\ignore{

\item ({\bf X points})
In our class discussion of the Count-Min sketch, we focused on the {\em strict turnstile} model. Here, each update increments or decrements a coordinate of a vector $\mathbf{x} \in \R^n$ in such a way that $\mathbf{x}_i \geq 0$ for all $i \in [n]$ always.

Now, consider the more general {\em turnstile} model where each update can increment or decrement a coordinate of $\mathbf{x}$ arbitrarily. In this problem, you will modify the Count-Min sketch so that it works in the turnstile model.

Recall the notation used. The Count-Min sketch uses a table $C$ of width $w$ and depth $d$. It maintains $d$ hash functions $h_1, \dots, h_d: [n] \to [w]$ where each $h_\ell$ is pairwise independent. The Count-Min sketch maintains that for all $\ell \in [d]$ and $j \in [w]$:
$$C[\ell, j] = \sum_{i: h_\ell(i)  = j} \mathbf{x}_i$$
\begin{enumerate}
\item[(a)]
Argue that for any $i \in [n]$ and $\ell \in [d]$, $$\EX[|C[\ell, h_\ell(i)] - \mathbf{x}_i|] \leq \frac{1}{w} \|\mathbf{x}\|_1$$ 
\item[(b)]
Let $\hat{\mathbf{x}}_i = \text{median}(\{C[\ell, h_\ell(i)] : \ell \in [d]\})$. Suppose $w = 2/\eps$ and $d = O(\log \delta^{-1})$. Argue that:
$$\Pr[|\hat{\mathbf{x}}_i - \mathbf{x}_i| > 3 \eps \|\mathbf{x}\|_1] < \delta$$
\end{enumerate}
}

\end{enumerate}
\end{document}