\documentclass[11pt]{article}
\usepackage{graphicx}
\usepackage{algorithm,algorithmic}
\usepackage{amsmath, amssymb, amsthm}
\usepackage{url}
\usepackage{fullpage, prettyref}
\usepackage{pstricks}
\usepackage{boxedminipage}
\usepackage{hyperref}
\usepackage{wrapfig}
\usepackage{ifthen}
\newtheorem{theorem}{Theorem}
\newtheorem{lemma}{Lemma}
\newtheorem{claim}{Claim}
\newtheorem{corollary}{Corollary}
\newtheorem{definition}{Definition}
\newtheorem{proposition}{Proposition}
\newtheorem{fact}{Fact}
\newtheorem{example}{Example}
\newtheorem{exercise}{Exercise}
\newtheorem{assumption}{Assumption}
\newtheorem{observation}{Observation}

\newcommand{\comment}[1]{\textsl{\small[#1]}\marginpar{\tiny\textsc{To Do!}}}
\newcommand{\ignore}[1]{}

\def\eps{\varepsilon}
\def\bar{\overline}
\def\floor#1{\lfloor {#1} \rfloor}
\def\ceil#1{\lceil {#1} \rceil}
\def\script#1{\mathcal{#1}}

\def\plus{{\tt (+)}}
\def\2plus{{\tt (++)}}
\def\3plus{{\tt (+++)}}
\def\4plus{{\tt (++++)}}
\def\5plus{{\tt (+++++)}}

\def\opt{{\tt opt}}
\def\alg{{\tt alg}}

\newenvironment{proofof}[1]{\smallskip\noindent{\bf Proof of #1:}}%
        {\hspace*{\fill}$\Box$\par}

\setlength{\oddsidemargin}{0pt}
\setlength{\evensidemargin}{0pt}
\setlength{\textwidth}{6.5in}
\setlength{\topmargin}{0in}
\setlength{\textheight}{8.5in}
\newlength{\algobox}
\setlength{\algobox}{6.5in}

\newcommand{\cA}{{\cal A}}
\newcommand{\cB}{\mathcal{B}}
\newcommand{\cC}{{\cal C}}
\newcommand{\cD}{\mathcal{D}}
\newcommand{\cE}{{\cal E}}
\newcommand{\cF}{\mathcal{F}}
\newcommand{\cG}{\mathcal{G}}
\newcommand{\cH}{{\cal H}}
\newcommand{\cI}{{\mathcal{I}}}
\newcommand{\cJ}{{\cal J}}
\newcommand{\cL}{{\cal L}}
\newcommand{\cM}{{\cal M}}
\newcommand{\cP}{\mathcal{P}}
\newcommand{\cQ}{\mathcal{Q}}
\newcommand{\cR}{{\cal R}}
\newcommand{\cS}{\mathcal{S}}
\newcommand{\cT}{{\cal T}}
\newcommand{\cU}{{\cal U}}
\newcommand{\cV}{{\cal V}}
\newcommand{\cX}{{\cal X}}


\newcommand{\R}{\mathbb R}
\newcommand{\N}{\mathbb N}
\newcommand{\F}{\mathbb F}
\newcommand{\Z}{{\mathbb Z}}
\renewcommand{\eps}{\varepsilon}
\newcommand{\lam}{\lambda}
\newcommand{\sgn}{\mathrm{sgn}}
\newcommand{\poly}{\mathrm{poly}}
\newcommand{\polylog}{\mathrm{polylog}}
\newcommand{\littlesum}{\mathop{{\textstyle \sum}}}
\newcommand{\half}{{\textstyle \frac12}}
\newcommand{\la}{\langle}
\newcommand{\ra}{\rangle}
\newcommand{\wh}{\widehat}
\newcommand{\wt}{\widetilde}
\newcommand{\calE}{{\cal E}}
\newcommand{\calL}{{\cal L}}
\newcommand{\calF}{{\cal F}}
\newcommand{\calW}{{\cal W}}
\newcommand{\calH}{{\cal H}}
\newcommand{\calN}{{\cal N}}
\newcommand{\calO}{{\cal O}}
\newcommand{\calP}{{\cal P}}
\newcommand{\calV}{{\cal V}}
\newcommand{\calS}{{\cal S}}
\newcommand{\calT}{{\cal T}}
\newcommand{\calD}{{\cal D}}
\newcommand{\calC}{{\cal C}}
\newcommand{\calX}{{\cal X}}
\newcommand{\calY}{{\cal Y}}
\newcommand{\calZ}{{\cal Z}}
\newcommand{\calA}{{\cal A}}
\newcommand{\calB}{{\cal B}}
\newcommand{\calG}{{\cal G}}
\newcommand{\calI}{{\cal I}}
\newcommand{\calJ}{{\cal J}}
\newcommand{\calR}{{\cal R}}
\newcommand{\calK}{{\cal K}}
\newcommand{\calU}{{\cal U}}
\newcommand{\barx}{\overline{x}}
\newcommand{\bary}{\overline{y}}

\newcommand{\ba}{\boldsymbol{a}}
\newcommand{\bb}{\boldsymbol{b}}
\newcommand{\bp}{\boldsymbol{p}}
\newcommand{\bt}{\boldsymbol{t}}
\newcommand{\bv}{\boldsymbol{v}}
\newcommand{\bx}{\boldsymbol{x}}
\newcommand{\by}{\boldsymbol{y}}
\newcommand{\bz}{\boldsymbol{z}}
\newcommand{\br}{\boldsymbol{r}}
\newcommand{\bh}{\boldsymbol{h}}

\newcommand{\bA}{\boldsymbol{A}}
\newcommand{\bD}{\boldsymbol{D}}
\newcommand{\bG}{\boldsymbol{G}}

\newcommand{\bR}{\boldsymbol{R}}
\newcommand{\bS}{\boldsymbol{S}}
\newcommand{\bX}{\boldsymbol{X}}
\newcommand{\bY}{\boldsymbol{Y}}
\newcommand{\bZ}{\boldsymbol{Z}}

\newcommand{\NN}{\mathbb{N}}
\newcommand{\RR}{\mathbb{R}}

\newcommand{\abs}[1]{\left\lvert #1 \right\rvert}
\newcommand{\norm}[1]{\left\lVert #1 \right\rVert}
\renewcommand{\ceil}[1]{\lceil#1\rceil}
\newcommand{\Exp}{\EX}
\renewcommand{\floor}[1]{\lfloor#1\rfloor}

\newcommand{\EX}{\hbox{\bf E}}
\newcommand{\prob}{{\rm Prob}}

\newcommand{\gset}{Y}
\newcommand{\gcol}{{\cal Y}}

%% Hyper-linked References
\newcommand{\Sec}[1]{\hyperref[sec:#1]{\S\ref*{sec:#1}}} %section
\newcommand{\Eqn}[1]{\hyperref[eq:#1]{(\ref*{eq:#1})}} %equation
\newcommand{\Fig}[1]{\hyperref[fig:#1]{Fig.\,\ref*{fig:#1}}} %figure
\newcommand{\Tab}[1]{\hyperref[tab:#1]{Tab.\,\ref*{tab:#1}}} %table
\newcommand{\Thm}[1]{\hyperref[thm:#1]{Theorem\,\ref*{thm:#1}}} %theorem
\newcommand{\Fact}[1]{\hyperref[fact:#1]{Fact\,\ref*{fact:#1}}} %fact
\newcommand{\Lem}[1]{\hyperref[lem:#1]{Lemma\,\ref*{lem:#1}}} %lemma
\newcommand{\Prop}[1]{\hyperref[prop:#1]{Prop.~\ref*{prop:#1}}} %property
\newcommand{\Cor}[1]{\hyperref[cor:#1]{Corollary~\ref*{cor:#1}}} %corollary
\newcommand{\Conj}[1]{\hyperref[conj:#1]{Conjecture~\ref*{conj:#1}}} %conjecture
\newcommand{\Def}[1]{\hyperref[def:#1]{Definition~\ref*{def:#1}}} %definition
\newcommand{\Alg}[1]{\hyperref[alg:#1]{Alg.~\ref*{alg:#1}}} %algorithm
\newcommand{\Ex}[1]{\hyperref[ex:#1]{Ex.~\ref*{ex:#1}}} %example
\newcommand{\Clm}[1]{\hyperref[clm:#1]{Claim~\ref*{clm:#1}}} %example

\newcommand{\Sol}{{\bf Solution sketch:} }

\usepackage{MnSymbol,wasysym}

\newcommand{\el}{\ensuremath{\ell}}


\def\Poi{{\sf Poi}}

\begin{document}

\title{E0234: Solution Sketch of Assignment 4}
\author{}
\date{}
\maketitle
It is highly recommended you do not google for the answers to the questions below. You can discuss with your friends, but then mention that in your submission.
The writing should solely be your own.

\begin{enumerate}
\item  {\bf (Filling up the gaps from class presentation.)}
\begin{enumerate}
\item For $X \sim \Poi(\lambda)$, prove $\mathbf{Exp}[X] = \mathbf{Var}[X] = \lambda$. Given $X\sim\Poi(\lambda)$ and $Y\sim \Poi(\mu)$ which are independent, prove
$X+Y \sim \Poi(\lambda+\mu)$.

\Sol 

$\mathbf{Exp}[X] = e^{-\lambda} \sum_{i=1}^\infty i \frac{\lambda^i}{i!} = e^{-\lambda} \lambda \sum_{i=1}^\infty i \frac{\lambda^{i-1}}{i!} = \lambda$

$\mathbf{Exp}[X^2] = e^{-\lambda} \sum_{i=1}^\infty i^2 \frac{\lambda^i}{i!} = \lambda(\lambda-1) $

$\Rightarrow \mathbf{Var}[X] = \lambda$

$\mathbf{Pr}[X+Y = k] = e^{-\lambda-\mu} \sum_{i=0}^k \frac{\lambda^i \mu^{k-i}}{i! (k-i)!} = e^{-\lambda-\mu} \frac{(\lambda+\mu)^k}{k!}$

\item  Let $\calE$ be an event in the balls-and-bins experiment whose probability is either monotonically increasing or decreasing with the number of balls thrown. 
Let $\calE'$ be the corresponding event when each bin independently has $Y_i \sim \Poi(m/n)$ many balls.

Prove that $\Pr[\calE] \leq 2\Pr[\calE']$. {\bf Hint:} You can use without proof that for $X\sim \Poi(\lambda)$, when $\lambda$ is a positive integer, $\mathbf{Med}(X) = \lambda$.

\Sol

Let $Z = \sum_{i=1}^n Y_i$. Then we know the following:
Consider the case when $Pr[\calE]$ is monotonically increasing with $m$ (the other case is similar).
\begin{align*}
 Pr[\calE] &= Pr[\calE' | Z=m] &\\
 &\le Pr[\calE' | Z\ge m] &\text{[since Pr[$\calE$] is monotonically increasing with $m$]}\\
 &= \frac{Pr[\calE' \cap Z\ge m]}{Pr[Z\ge m]} &\\
 &= 2 Pr[\calE' \cap Z\ge m] &\text{[since $m$ is median of $Z$]}\\
 &\le 2Pr[\calE']&
\end{align*}


\item Recall some notation from class: $Y_i \sim \Poi(m/n)$ and $Z = \sum_i Y_i$. Let $\calE$ be an event which is monotonically decreasing with $Z$; that is, $\Pr[\calE~|Z=k] \geq \Pr[\calE~|Z=k']$ if $k < k'$. Suppose $\Pr[\calE|Z=a] - \Pr[\calE|Z=b] \leq \delta$ for some non-negative integers $a < b$. 

Prove
\[
\Big| \Pr[\calE~|a \leq Z \leq b ]  - \Pr[\calE~|Z = c] \Big| \leq \delta, ~~~~~\forall a\leq c\leq b.
\]

\Sol

\begin{align*}
 \Pr[\calE~|a\leq Z\leq b] &= \frac{\Pr[\calE \cap a\leq Z\leq b]}{\Pr[a\leq Z\leq b]}\\
 &= \frac{\sum_{i=a}^b \Pr[\calE \cap Z=i]}{\sum_{i=a}^b \Pr[Z=i]}
\end{align*}

Now we know:

\[ \frac{\Pr[\calE \cap Z=a]}{\Pr[Z=a]} \geq \frac{\Pr[\calE \cap Z=a+1]}{\Pr[Z=a+1]}\geq \cdots \frac{\Pr[\calE \cap Z=b]}{\Pr[Z=b]}\]

Hence, we have:

\[ \Pr[\calE ~| Z=b] \leq \Pr[\calE ~| a \leq Z\leq b] \leq \Pr[\calE ~| Z=a] \]

Therefore,

\[
\Rightarrow \Big| \Pr[\calE~|a \leq Z \leq b ]  - \Pr[\calE~|Z = c] \Big| \leq \delta, ~~~~~\forall a\leq c\leq b.
\]

\end{enumerate}


\item Consider the coupon collector setting. Let $\tau$ be the (random) time when you observe $k$ copies of every coupon where $k$ is a fixed integer $> 1$.
Prove that for any real constant $c$,
\[\lim_{n\to \infty} \Pr[\tau > n\left(\ln n + (k-1)\ln\ln n + c\right)] = 1 - e^{-e^{-c}}\]

\Sol

Let $\calE'$ be the event that there exists at least one coupon that has been seen at most $k-1$ times in $n\left(\ln n + (k-1)\ln\ln n + c\right)$ draws.

\begin{align*}
 \Pr[\calE'] &= 1 - \Pr[\forall i, Y_i \ge k]\\
	     &= 1 - \left(1 - e^{-\frac{m}{n}} \sum_{i=0}^{k-1} \frac{(m/n)^i}{i!} \right)^n\\
	     &\ge 1 - \left(1- e^{-\frac{m}{n}} \frac{(m/n)^{k-1}}{(k-1)!} \right)^n\\
	     &\ge 1 - \left(1 - \frac{e^{-c}}{n} \frac{1}{(\ln n)^{k-1}} (\ln n)^{k-1}	     \right)^n\\
	     &= 1 - e^{e^{-c}}
\end{align*}

The rest of the calculation is similar to the one done in class.


\item Consider $n$ balls thrown into $n$ bins. What can you say about the $k$th largest load?
More precisely if the load profile is $(X_1,\ldots,X_n)$, define
$L_k := \textrm{$k$th-largest}(X_1,\ldots,X_n).$
Express the function $f(n,k)$ such that $L_k = \Theta(f(n,k))$ whp. %From what we did in class, we get $p(n,1) = \log n/\log\log n$.
Start with $k=2$. 

\Sol

\begin{align*}
 \Pr[L_k > t] &\le {n\choose k} {n\choose t} {(n-t)\choose t} \cdots {(n-(k-1)t)\choose t} \frac{1}{n^{kt}}\\
 &\approx \frac{n^{(t+1)k}}{t!^k n^{kt}}\\
 &= \left( \frac{n}{t!} \right)^k
\end{align*}

Choose $t = (2+\frac{1}{k}) \frac{\ln n}{\ln \ln n}$.


\item
Consider a counter that receives packets arriving on a stream. The counter is initialized to $0$. If the counter is currently at $i$ and a new packet arrives, then the counter is incremented with probability $2^{-i}$ (and otherwise remains unchanged). Suppose $n$ packets arrive in total, and let $X_n$ be the random variable denoting the counter value at the end of the stream.
\begin{enumerate}
\item[(a)]
Show that $\mathbf{Exp}[2^{X_n}] = n+1$. 

\Sol

Inductive step: $\mathbf{Exp}[2^{X_n}] = \mathbf{Exp}[2^{X_{n-1}+1} 2^{-X_{n-1}}] + \mathbf{Exp}[2^{X_{n-1}} (1-2^{-X_{n-1}})] = 1 + \mathbf{Exp}[2^{X_{n-1}}] = n+1$
\item[(b)]
Show that $\mathbf{Var}[2^{X_n}] = \frac{n(n-1)}{2}$.

\Sol

\begin{align*}
 \mathbf{Var}[2^{X_n}] &= \mathbf{Exp}[2^{2X_n}] - \mathbf{Exp}[2^{X_n}]^2\\
 &= \mathbf{Exp}\left[2^{2X_{n-1}} 2 2^{-X_{n-1}} + 2^{2X_{n-1}}\left(1- 2^{-X_{n-1}}\right)\right] - (n-1)^2\\
 &= \mathbf{Exp}[2^{2X_{n-1}}] - n^2 + 3\mathbf{Exp}[2^{X_{n-1}}] - 2n - 1\\
 &= \frac{(n-1)(n-2)}{2} + n-1\\
 &=\frac{n(n-1)}{2}
\end{align*}

\end{enumerate}

\noindent \textbf{Hint}: Use induction on $n$.
\end{enumerate}

\end{document}