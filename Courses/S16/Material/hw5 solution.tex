\documentclass[11pt]{article}
\usepackage{graphicx}
\usepackage{algorithm,algorithmic}
\usepackage{amsmath, amssymb, amsthm}
\usepackage{url}
\usepackage{fullpage, prettyref}
\usepackage{pstricks}
\usepackage{boxedminipage}
\usepackage{hyperref}
\usepackage{wrapfig}
\usepackage{ifthen}
\newtheorem{theorem}{Theorem}
\newtheorem{lemma}{Lemma}
\newtheorem{claim}{Claim}
\newtheorem{corollary}{Corollary}
\newtheorem{definition}{Definition}
\newtheorem{proposition}{Proposition}
\newtheorem{fact}{Fact}
\newtheorem{example}{Example}
\newtheorem{exercise}{Exercise}
\newtheorem{assumption}{Assumption}
\newtheorem{observation}{Observation}

\newcommand{\comment}[1]{\textsl{\small[#1]}\marginpar{\tiny\textsc{To Do!}}}
\newcommand{\ignore}[1]{}

\def\eps{\varepsilon}
\def\bar{\overline}
\def\floor#1{\lfloor {#1} \rfloor}
\def\ceil#1{\lceil {#1} \rceil}
\def\script#1{\mathcal{#1}}

\def\plus{{\tt (+)}}
\def\2plus{{\tt (++)}}
\def\3plus{{\tt (+++)}}
\def\4plus{{\tt (++++)}}
\def\5plus{{\tt (+++++)}}

\def\opt{{\tt opt}}
\def\alg{{\tt alg}}

\newenvironment{proofof}[1]{\smallskip\noindent{\bf Proof of #1:}}%
        {\hspace*{\fill}$\Box$\par}

\setlength{\oddsidemargin}{0pt}
\setlength{\evensidemargin}{0pt}
\setlength{\textwidth}{6.5in}
\setlength{\topmargin}{0in}
\setlength{\textheight}{8.5in}
\newlength{\algobox}
\setlength{\algobox}{6.5in}

\newcommand{\cA}{{\cal A}}
\newcommand{\cB}{\mathcal{B}}
\newcommand{\cC}{{\cal C}}
\newcommand{\cD}{\mathcal{D}}
\newcommand{\cE}{{\cal E}}
\newcommand{\cF}{\mathcal{F}}
\newcommand{\cG}{\mathcal{G}}
\newcommand{\cH}{{\cal H}}
\newcommand{\cI}{{\mathcal{I}}}
\newcommand{\cJ}{{\cal J}}
\newcommand{\cL}{{\cal L}}
\newcommand{\cM}{{\cal M}}
\newcommand{\cP}{\mathcal{P}}
\newcommand{\cQ}{\mathcal{Q}}
\newcommand{\cR}{{\cal R}}
\newcommand{\cS}{\mathcal{S}}
\newcommand{\cT}{{\cal T}}
\newcommand{\cU}{{\cal U}}
\newcommand{\cV}{{\cal V}}
\newcommand{\cX}{{\cal X}}


\newcommand{\R}{\mathbb R}
\newcommand{\N}{\mathbb N}
\newcommand{\F}{\mathbb F}
\newcommand{\Z}{{\mathbb Z}}
\renewcommand{\eps}{\varepsilon}
\newcommand{\lam}{\lambda}
\newcommand{\sgn}{\mathrm{sgn}}
\newcommand{\poly}{\mathrm{poly}}
\newcommand{\polylog}{\mathrm{polylog}}
\newcommand{\littlesum}{\mathop{{\textstyle \sum}}}
\newcommand{\half}{{\textstyle \frac12}}
\newcommand{\la}{\langle}
\newcommand{\ra}{\rangle}
\newcommand{\wh}{\widehat}
\newcommand{\wt}{\widetilde}
\newcommand{\calE}{{\cal E}}
\newcommand{\calL}{{\cal L}}
\newcommand{\calF}{{\cal F}}
\newcommand{\calW}{{\cal W}}
\newcommand{\calH}{{\cal H}}
\newcommand{\calN}{{\cal N}}
\newcommand{\calO}{{\cal O}}
\newcommand{\calP}{{\cal P}}
\newcommand{\calV}{{\cal V}}
\newcommand{\calS}{{\cal S}}
\newcommand{\calT}{{\cal T}}
\newcommand{\calD}{{\cal D}}
\newcommand{\calC}{{\cal C}}
\newcommand{\calX}{{\cal X}}
\newcommand{\calY}{{\cal Y}}
\newcommand{\calZ}{{\cal Z}}
\newcommand{\calA}{{\cal A}}
\newcommand{\calB}{{\cal B}}
\newcommand{\calG}{{\cal G}}
\newcommand{\calI}{{\cal I}}
\newcommand{\calJ}{{\cal J}}
\newcommand{\calR}{{\cal R}}
\newcommand{\calK}{{\cal K}}
\newcommand{\calU}{{\cal U}}
\newcommand{\barx}{\overline{x}}
\newcommand{\bary}{\overline{y}}

\newcommand{\ba}{\boldsymbol{a}}
\newcommand{\bb}{\boldsymbol{b}}
\newcommand{\bp}{\boldsymbol{p}}
\newcommand{\bt}{\boldsymbol{t}}
\newcommand{\bv}{\boldsymbol{v}}
\newcommand{\bx}{\boldsymbol{x}}
\newcommand{\by}{\boldsymbol{y}}
\newcommand{\bz}{\boldsymbol{z}}
\newcommand{\br}{\boldsymbol{r}}
\newcommand{\bh}{\boldsymbol{h}}

\newcommand{\bA}{\boldsymbol{A}}
\newcommand{\bD}{\boldsymbol{D}}
\newcommand{\bG}{\boldsymbol{G}}

\newcommand{\bR}{\boldsymbol{R}}
\newcommand{\bS}{\boldsymbol{S}}
\newcommand{\bX}{\boldsymbol{X}}
\newcommand{\bY}{\boldsymbol{Y}}
\newcommand{\bZ}{\boldsymbol{Z}}

\newcommand{\NN}{\mathbb{N}}
\newcommand{\RR}{\mathbb{R}}

\newcommand{\abs}[1]{\left\lvert #1 \right\rvert}
\newcommand{\norm}[1]{\left\lVert #1 \right\rVert}
\renewcommand{\ceil}[1]{\lceil#1\rceil}
\newcommand{\Exp}{\EX}
\renewcommand{\floor}[1]{\lfloor#1\rfloor}

\newcommand{\EX}{\hbox{\bf E}}
\newcommand{\prob}{{\rm Prob}}

\newcommand{\gset}{Y}
\newcommand{\gcol}{{\cal Y}}

%% Hyper-linked References
\newcommand{\Sec}[1]{\hyperref[sec:#1]{\S\ref*{sec:#1}}} %section
\newcommand{\Eqn}[1]{\hyperref[eq:#1]{(\ref*{eq:#1})}} %equation
\newcommand{\Fig}[1]{\hyperref[fig:#1]{Fig.\,\ref*{fig:#1}}} %figure
\newcommand{\Tab}[1]{\hyperref[tab:#1]{Tab.\,\ref*{tab:#1}}} %table
\newcommand{\Thm}[1]{\hyperref[thm:#1]{Theorem\,\ref*{thm:#1}}} %theorem
\newcommand{\Fact}[1]{\hyperref[fact:#1]{Fact\,\ref*{fact:#1}}} %fact
\newcommand{\Lem}[1]{\hyperref[lem:#1]{Lemma\,\ref*{lem:#1}}} %lemma
\newcommand{\Prop}[1]{\hyperref[prop:#1]{Prop.~\ref*{prop:#1}}} %property
\newcommand{\Cor}[1]{\hyperref[cor:#1]{Corollary~\ref*{cor:#1}}} %corollary
\newcommand{\Conj}[1]{\hyperref[conj:#1]{Conjecture~\ref*{conj:#1}}} %conjecture
\newcommand{\Def}[1]{\hyperref[def:#1]{Definition~\ref*{def:#1}}} %definition
\newcommand{\Alg}[1]{\hyperref[alg:#1]{Alg.~\ref*{alg:#1}}} %algorithm
\newcommand{\Ex}[1]{\hyperref[ex:#1]{Ex.~\ref*{ex:#1}}} %example
\newcommand{\Clm}[1]{\hyperref[clm:#1]{Claim~\ref*{clm:#1}}} %example

\newcommand{\Sol}{{\bf Solution sketch:} }

\usepackage{MnSymbol,wasysym}

\newcommand{\el}{\ensuremath{\ell}}


\def\Poi{{\sf Poi}}

\begin{document}
\def\Exp{\mathbf{Exp}}
\def\Var{\mathbf{Var}}
\title{E0234: Assignment 5 Solutions}
\author{}
\date{}
\maketitle
It is highly recommended you do not google for the answers to the questions below. You can discuss with your friends, but then mention that in your submission.
The writing should solely be your own.

\begin{enumerate}

\item Let $X_1,\ldots , X_n$ be independent indicator random variables, and let $p_i := \Pr[X_i = 1]$.
Let $U$ and $L$ be any positive reals such that $U \geq \sum_{i=1}^n p_i \geq L$. Let $X = \sum_{i=1}^n X_i$.
Prove
\begin{enumerate}
	\item $\Pr\left[X \geq (1+\delta)U\right] \leq e^{-U((1+\delta)\ln(1+\delta) -\delta)}$
	\item $\Pr\left[X \leq (1-\delta)L\right] \leq e^{-L\delta^2/2}$
\end{enumerate}

\Sol

Follows from the proof of Theorem 4.2 and 4.2 in MR.

\item  We are given a collection of $n$ sets $S_1,\ldots,S_n$ where each $S_i$ is a subset of $U = \{1,2,\ldots,n\}$.
A $2$-coloring is an assignment  $c:U \to \{-1,+1\}$ to the elements of $U$.
The {\em discrepancy} of a $2$-coloring is $\max_{j=1}^n \sum_{i\in S_j} c(i)$. 
What can you say about the discrepancy of a random 
coloring which independently assigns each element $+1$ or $-1$ with probability $1/2$? 
In particular, give a confidence bound (in terms of $n$) in which the discrepancy lies with probability $> 99\%$.

\Sol

For $\{1, -1\}$ random variable, we get the following from the Chernoff bound.

\begin{align*}
 \Pr[c(S_i) \ge \Delta] \le  e^{-\frac{\Delta^2}{|S_i|}} \le  e^{-\frac{\Delta^2}{n}}
\end{align*}

Using union bound, we have the following.

\[ \Pr[\exists i, |c(S_i)| \ge \Delta] \le n e^{-\frac{\Delta^2}{n}}\]

Hence, $\max_{j=1}^n \sum_{i\in S_j} c(i)$ is at most $O(\sqrt{n\log n})$ within $99\%$ probability.


\item 
\begin{enumerate}
	\item If $Z$ is a random variable with $\Exp[Z] = 0$ and $|Z| \leq 1$ with probability $1$, then prove $Z$ is $1$-subgaussian.
	
	\Sol 
	
	\begin{align*}
	 \Exp[e^{tZ}] &= \Exp[1 + tZ + \frac{t^2 Z^2}{2!} + \ldots]\\
	 &\le 1 + \frac{t^2}{2!} + \frac{t^4}{4!} + \ldots\\
	 &= e^{t^2/2}
	\end{align*}


	\item If $Z$ is a $\sigma$-subgaussian random variable, prove that $\Exp[Z] = 0$ and $\Var[Z] \leq \sigma^2$.
	
	\Sol
	
	\begin{align*}
	  &\Exp[e^{tZ}] \le e^{\frac{t^2 \sigma^2}{2}}&\\
	  \Rightarrow &1 + t\Exp[Z] + \frac{t^2}{2}\Exp[Z^2] + \frac{t^3}{3!}\Exp[Z^3] \ldots \le e^{\frac{t^2 \sigma^2}{2}}&\\
	  \Rightarrow &\Exp[Z] + t\Exp[Z^2] + \frac{t^2}{3}\Exp[Z^3] + \ldots \le 2te^{\frac{t^2 \sigma^2}{2}} &[\text{Differentiating w.r.t }t]\\
	  \Rightarrow &\Exp[Z] \le 0 &[\text{Putting }t=0]\\
	  \Rightarrow &\Exp[Z] = 0 &[\text{Since Z is subgaussian, -Z is subgaussian too}]
	\end{align*}

\end{enumerate}

\item {\bf Johnson-Lindenstrauss Lower Bound.} Consider the following $(n+1)$ points in $n$ dimensions: $\{0,e_1,\ldots,e_n\}$
where $e_i$ is the $i$th unit vector with $1$ in the $i$th position and $0$ elsewhere. 
Consider a mapping of these $(n+1)$ vectors to $k$ dimensions such that the pairwise distances are preserved to a $(1\pm \eps)$-multiplicative factor.
What is the best {\em lower bound} you can prove on $k$? 

\Sol See \href{https://courses.cs.washington.edu/courses/cse522/14sp/lectures/lect10.pdf}{here} and \href{https://courses.cs.washington.edu/courses/cse522/14sp/lectures/lect11.pdf}{here}.

\end{enumerate}

\end{document}