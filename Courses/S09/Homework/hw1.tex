\documentclass[10pt]{article}
\usepackage{hw}
\usepackage{amsthm}

\LectureNumber{1}
\LectureDate{May 21st, 2009}
\LectureTitle{}

\usepackage{amsmath}
\usepackage{latexsym}
\usepackage{amssymb}
\usepackage{fancybox}
\usepackage{psfig}


\newcommand{\ddate}[1]{\noindent {\marginpar{ \bf #1}} \newline \indent}

\def\ni{\noindent}

\setlength{\oddsidemargin}{0pt}
\setlength{\evensidemargin}{0pt}
\setlength{\textwidth}{6.0in}
\setlength{\topmargin}{0in}
\setlength{\textheight}{9in}

\begin{document}
\MakeScribeTop
\begin{enumerate}
\item Given precedence constraints for a machine scheduling problem as a directed graph, how would you check if the graph is acyclic (that is the precedence constraints are consistent)? What is the best time complexity (in the big-Oh notation) you can come up with?

\item 
Suppose $\gamma_1$ and $\gamma_2$ are regular performance measures. Which of the following are regular? If you say a measure is regular, prove it. If you say otherwise, give an example to show that it is not.
\begin{itemize}
\item $w_1\gamma_1 + w_2\gamma_2$, for $w_1,w_2 \ge 0$ .
\item $\gamma_1 - \gamma_2$ .
\item $w_1\gamma_1 - w_2\gamma_2$, for $w_1 > w_2 \ge 0$ .
\item $e^{c\gamma_1}$, for some $c > 0$. Here $e$ is the base of the natural logarithm, $2.718\ldots$ .
\item $\gamma_1/\gamma_2$ .
\end{itemize}

\item 
Given a schedule for a scheduling problem, let $N_u(t)$ denote the number of unfinished jobs at time $t$. Therefore, $N_u(0) = n$ and $N_u(C_{max}) = 0$. Consider the performance measure
$$\overline{N_u} := \frac{1}{C_{max}} ~\int^{C_{max}}_{0} N_u(t)dt$$

Express this performance measure as a function of performance measures we studied in class. Is this performance measure regular? Why or why not?


\item
Given an instance of a machine scheduling problem $(J ~|~|~ \gamma)$ where $\gamma$ is a regular performance measure, show that there exists an active schedule which is optimal. 

\item 
In the {\em Traveling Salesman Problem} (TSP), we have a salesman who needs to start at his office at city $0$ and visit $n$ cities and come back to his office. The distance between city $i$ and $j$ for $0 \le i\neq j \le n$ is given as $d_{ij}$. The problem is to find a tour of the salesman with minimum total distance. Show that this problem is equivalent to a scheduling problem in the $(\alpha ~|~ \beta ~|~ \gamma)$ notation. (By equivalence, we mean solving one problem would imply the solution to another.)

\item 
Consider the following shop scheduling problem with $2$ machines and $4$ jobs.
\begin{center}
\begin{tabular}{|c||c|c|c|c|c|c|c|c|}\hline
Jobs & 1 & 2 & 3 & 4 \\ \hline
$p_{1j}$ & 8 & 6 & 4 & 12 \\
$p_{2j}$ & 4 & 9 & 10 & 6 \\
\hline
\end{tabular}
\end{center}
\vspace{0.1in}
Consider the two problems, $(F2~|~|~C_{max})$ and $(O2 ~|~|~C_{max})$,
given the above data. By problem $4$ above we know that there exists an optimal 
active schedule. How many active schedules are there for the above problem?
Find optimal schedules for the above two problems and {\em argue}  that your schedules are optimal. Try to make your argument as rigorous as you can. 
Arguments such as ``I have tried all possible schedules'' will not be considered rigorous and are not recommended.
(Hint: Try to argue that $C_{max}$ of any schedule must be at least some value depending only on the processing times and then construct the schedule)
\end{enumerate}
\end{document}

